\documentclass[../../main.tex]{subfiles}
\begin{document}

\section[First order correction]{First order correction $v_1$}

We have obtained a policy function $\varphi(p)$. Let the deterministic optimal emission schedule be \begin{equation}
    e = \varphi_0(x, c) = \varphi\Big(- \partial_c v_0(x, c)\Big).
\end{equation}

The first oder correction satisfies the partial differential equation \begin{equation}
    \begin{split}
        \rho \ v_1(x, c) = \ &\partial_x v_1(x, c) \ \Big(\mu(x) + C_0 + C_1 \log(c / c_p)\Big) \\
        + \ &\partial_c v_1(x, c) \Big( \varphi_0(x, c) - \delta c \Big) \\
        + \ &\sigma^2 \ \partial_{x, x} v_0(x, c).
    \end{split}
\end{equation}

Furthermore, we can derive boundaries condition on the nullclines. We define the nullcline of $x$ to be the curve \begin{equation}
    \psi^x = \left(x, c_p \exp\left(-\frac{\mu(x) + C_0}{C_1}\right) \right)
\end{equation}. Along $\psi^x$ we have $\dot x = 0$, hence we obtain the differential equation \begin{equation}
    \rho \ v_1(\psi^x) = \partial_c v_1(\psi^x) \Big( \varphi_0(\psi^x) - \delta \ \psi^x_2 \Big) + \sigma^2 \ \partial_{x, x} v_0(\psi^x).
\end{equation}

In a similar way, we can compute the nullcline $\psi^c$ along which $\dot{c} = 0$. Along this curve, we have the partial differential equation \begin{equation}
    \rho \ v_1(\psi^c) = \partial_x v_1(\psi^c) \ \Big(\mu(\psi^c_1) + C_0 + C_1 \log(\psi^c_2 / c_p)\Big) + \sigma^2 \ \partial_{x, x} v_0(\psi^c).
\end{equation}

Finally, we obtain an initial condition by noticing that at the steady states $\psi^c \cap \psi^x$ we have \begin{equation}
    \rho \ v_1(\psi^c \cap \psi^x) = \sigma^2 \ \partial_{x, x} v_0(\psi^c \cap \psi^x).
\end{equation}


\end{document}