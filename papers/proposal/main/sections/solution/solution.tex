\documentclass[../../main.tex]{subfiles}
\begin{document}

We solve the Hamilton-Bellman-Jacobi equation (\ref{eq:hbj}) using the method proposed by \citein{grass_small-noise_2015}. In particularly, rewriting the noise term in the temperature dynamics as $\sigma_x^2 = \varepsilon \sigma^2$, we compute a small $\varepsilon$ approximation of the value function $v$. We assume that there exist an asymptotic expansion of the value function as a power series of the noise scale $\varepsilon$, \begin{equation}
    v(x, c, \varepsilon) = v_0(x, c) + \varepsilon v_1(x, c) + \varepsilon^2 v_2(x, c) + \ldots
\end{equation} with $v_j$ being twice continuously differentiable functions. We then compute a second order approximation of $v$ by computing $v_0$, $v_1$, and $v_2$. These basis functions will then satisfy a system of ordinary differential equations, as opposed to the original partial differential Hamilton-Bellman-Jacobi equation. Particularly, we have that

\begin{align}
    \rho \ v_0(s) &= \H(s, \nabla v_0), \\
    \rho \ v_1(s) &= \nabla v_1(s) \cdot \nabla_p\H(s, \nabla v_0) + \sigma^2 \ \Dx^2 v_1(s), \\
    \rho \ v_2(s) &= \nabla v_2(s) \cdot \nabla_p\H(s,  \nabla v_0) + \sigma^2 \ \Dx^2 v_2(s) + \textbf{??}, \text{ and so on.}
\end{align}


\subsection{An Illustrative Example}

To illustrate our findings and for calibration purposes, we will use a linear quadratic model. We assume the benefit of emissions, in dollar terms, to be given by $u(e) = \beta_0 e - \frac{1}{2} \beta_1 e^2$ and temperature damages to be quadratic, $d(x) = \frac{1}{2}\gamma (x - x_p)^2$, where $x_p$ is the pre-industrial temperature level.

\end{document}