\documentclass[../../main.tex]{subfiles}
\begin{document}


Following \citein{pindyck_economic_2013} and \cite{hambel_optimal_2021}, I assume output $Y$ to be a linear function of capital $K$ , \begin{equation}
Y =  A(t) K,
\end{equation} where $A(t)$ denotes productivity. Output can be used for investment $I$, abatement expenditures $B$, or consumption $C$ \begin{equation}
    Y = I + B + C.
\end{equation} More conveniently we normalise abatement expenditures and investment by dividing by the capital stock $K$ \begin{equation}
    b \coloneqq B / K \text{ and } \; i \coloneqq I / K.
\end{equation}

Following Nordhaus [\textbf{cite}] we assume abatement expenditures to be proportional to output and quadratic in the emission control rate such that we can write \begin{equation}
    b(t, \varepsilon) = A(t) \frac{\omega(t)}{2} \varepsilon^2
\end{equation} where $\omega(t)$ is a decreasing exogenous coefficient capturing technological progress.

\subsection{Evolution of Capital and Climate Change}

Log-capital evolves as \begin{equation}
    \d \hat{k} \coloneq \frac{\d K}{K} = \Big( \phi(i) - \frac{\kappa_a}{2} \big(i + b\big)^2 - d(x) \Big) \d t
\end{equation}

where $d$ is a an increasing damage function of temperature and \begin{equation}
    \phi(i) = i - \delta_k.
\end{equation}

Letting $\chi \coloneqq C / Y$ be the consumption share of output we can write the investment ratio as \begin{equation}
    i = A(t) \left(1 - \chi - \frac{\omega(t)}{2} \varepsilon^2\right)
\end{equation}

Plugging this back into the evolution of log-capital we obtain \begin{equation}
    \frac{\d \hat k}{\d t} = A(t) \left(1 - \chi - \frac{\omega(t)}{2} \varepsilon^2\right) - \delta_k - \frac{\kappa_a \; A(t)^2}{2} (1 - \chi)^2 - d(x).
\end{equation} 

This equation allows us to decompose the growth of log-output $\hat y$ into three components. Namely, letting the productivity growth rate $\rho(t) \d t = \d \log A(t)$ and the capital adjustment in absence of climate change  \begin{align}
    \phi(t, \chi) = A(t) \big(1 - \chi\big) - \frac{\kappa_a \; A(t)^2}{2} (1 - \chi)^2 - \delta_k,
\end{align} we can write \begin{equation}
    \d \hat y = \Big(\rho(t) + \phi(t, \chi) - b(t, \varepsilon) - d(x) \Big) \d t.
\end{equation}

\iffalse
    \subsection{Old}
    To keep the analysis tractable and tease out the role of tipping points in the social cost of carbon, I assume a Ramsey economy always in equilibrium and with constant productivity. Namely, output is a Cobb-Douglas function of capital, $y = A(t) k^\alpha l^{1 - \alpha}$, where $A$ is total factor productivity. We assume the depreciation rate of capital to be $\delta_k$ and  $d(x) \in [0, 1]$ be an increasing and convex damage function of temperature. At all time, the remaining output is employed to either replace depreciated capital or for consumption \begin{equation}
        (1 - d(x)) y = \delta_k \ k + c.
    \end{equation}

    Finally, to link output $y$ and C0$_2$ emissions $e$ we use the Kaya identity \begin{equation}
        e = y \beta  \text{ where } \beta \coloneqq \underbrace{\beta_m}_{\substack{\text{Energy} \\ \text{intensity} \\ \text{of output}}} \underbrace{\beta_e,}_{\substack{\text{Emission} \\ \text{intensity} \\ \text{of energy}}}
    \end{equation} such that consumption can be written in terms of CO$_2$ emissions as \begin{equation}
        c(e, x) = (1 - d(x)) \; e \beta - \delta_k \; \left(\frac{e \beta}{A \; l^{(1 - \alpha)}}\right)^{1 / \alpha}. 
    \end{equation}

    I study the social planner problem, who seeks to maximise the expected discounted stream of utility of consumption \begin{equation}
        J(e; x_0, m_0) = \mathbb{E}_{x} \int^{\infty}_0 \mathrm{e}^{-\rho t} \ u\Big(c(e, x)\Big) \ \text{d} t,
    \end{equation} subject to the dynamics of temperature $\text{d}x$ and carbon concentration $\text{d}m$. To do so, she picks an emission path $e$, which I assume to be a continuously differentiable function of time taking values in an interval $\mathcal{E} = [0, \overline{e}]$. Notice that here we are assuming no possibility of negative emission. Furthermore, I assume benefits of emission to be increasing $u^\prime > 0$ and concave $u^{\prime \prime} < 0$. 

    For a given admissible state $(x, m)$, the value function is \begin{equation}
        v(x, m) = \sup_{e \in \mathcal{E}} J(e; x, m).
    \end{equation} 

    \begin{definition}
        Let the maximised current value Hamiltonian be \begin{equation}
            \begin{split}
                \H(x, m, \mathbf{\lambda}) = \ &u\Big(c\big(\varphi(\lambda_m, m), ƒx \big) \Big) \\
                + \ &\lambda_m \Big(\varphi(\lambda_m, m) - \delta m\Big) \\
                + \ &\lambda_x \Big(\mu(x) + S + M \log(m / m_p)\Big),
            \end{split}
        \end{equation} where \begin{equation}
            \varphi(\lambda_m, x) = \arg\max_{e \in \mathcal{E}} \Big\{ u(c(e, x)) + \lambda_m e \Big\},
        \end{equation} is the optimal emission schedule.
    \end{definition}

    \begin{proposition}
        The value function $v$ satisfies the Hamilton-Bellman-Jacobi partial differential equation, \begin{equation} \label{eq:hbj}
            \rho \ v(x, m) = \sigma^2_x \ \partial_{x x} v(x, m) + \H(x, m, \nabla v(x, m)),
        \end{equation} where $\nabla v$ is the vector of partial derivatives of $v$ with respect to $x$ and $c$.
    \end{proposition}

    \subsection{Optimal Emissions}

    In the rest of the paper and in the calibration, we take $u(c) = \log c$. In this case, optimal emissions $\varphi$ as a function of the shadow price of carbon concentration $\lambda_m$ and temperature $x$, solves the first order condition \begin{equation}
        \partial_e c \Big(\varphi(\lambda_m, x), x\Big) + \lambda_m \; c\Big(\varphi(\lambda_m, x), x\Big) = 0.
    \end{equation}
\fi


\end{document}