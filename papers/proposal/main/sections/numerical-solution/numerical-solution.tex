\documentclass[../../main.tex]{subfiles}
\begin{document}

Putting it all together, letting $X$ be the state vector, the value function at time $t$ satisfies \begin{equation}
    \begin{split}
        -\partial_t V_t &= G(t, X, V_t) \\
        &\coloneqq f(\control{\chi}(X, \nabla V_t), X, V_t) + \nabla V_t \cdot w(t, X, \control{\chi}(X), \control{\alpha}(X),  \nabla V_t) + \frac{\sigma^2}{2} \partial_{T}^2 V_t
    \end{split}
\end{equation}


\subsection{Post-transition phase}

We suppose that at some point in the future $\tau$, abatement is free and is exogenously imposed such that carbon concentration remains constant, $\gamma^b = \alpha$. I call this the \textit{post-transition} phase, which the following climate dynamics \begin{align}
    \d{m} &= 0, \\
    \epsilon \; \d T &= \mu(T, m) \; \d t + \sigma_T \; \d W \text{ and } \\
    \d y &= \big( \varrho + \phi(\chi) - \delta_k^p - d(T) \big) \; \d t.
\end{align}

Let $p(T, \overline{m})$ be the steady state density of temperature $T$ given the constant level of carbon concentration $\bar{m}$. It can be shown that \begin{equation}
    p(T, \overline{m}) \propto \exp\left(-\frac{\mu_m(\overline{m}) T + (1 - \lambda_1) S_0 T - \frac{\eta}{5} T^5 + S_0 (\lambda_1 - \lambda_2) \log(1 + \exp(T - T_i))}{\sigma_T^2 / 2\epsilon}\right).
\end{equation}

Ito's lemma on $d(T)$.
\begin{equation}
    \epsilon \; \d d(T) = \left(\mu(T, \overline{m}) d'(T) + \frac{\sigma^2_T}{2\epsilon} d''(T) \right) \d t + \sigma_T d'(T) \d W.
\end{equation}

\newpage

Let $\bar{V}(t, T, m, y)$ be the post-transition value function, which satisfies the Hamilton-Bellman-Jacobi equation \begin{equation}
    0 = \sup_{\chi} \Big\{f(\chi, y, \bar{V}) +  \partial_y \bar{V} \phi(\chi) \Big\} - \partial_y \bar{V} \Big(\delta^{p}_k + d(T)\Big) + \partial_T \bar{V} \; \frac{\mu(T, m)}{\epsilon}+ \partial^2_T \bar{V} \; \frac{\sigma^2_T}{2\epsilon}.
\end{equation}

\end{document}