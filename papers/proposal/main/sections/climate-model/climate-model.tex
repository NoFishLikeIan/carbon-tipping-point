\documentclass[../../main.tex]{subfiles}
\begin{document}

The climate model studied here is a simplified version of the one developed by \citein{mendez_investigating_2021}. The model consists in a coupled system of stochastic differential equations (SDEs) for atmospheric carbon concentration $c$ and average world temperature $x$. Carbon concentration evolves simply as \begin{equation} \label{eq:climate-model-c}
    \text{d}c = (e - \delta c) \text{d} t,
\end{equation} where $e$ is the level of emissions and $\delta$ is a natural decay rate. Temperature evolves as \begin{equation} \label{eq:climate-model-x}
    \frac{1}{\kappa} \text{d}x = \Big( \mu(x) + S + C \log(c / c_p) \Big) \text{d} t + \sigma^2 \text{d}w
\end{equation} where $c_p$ is the pre-industrial level of carbon concentration, $C_1$ is the direct forcing of CO$_2$, and $w$ is a Wiener process. The non-linearity in temperature dynamics is given by the function \begin{equation}
    \mu(x) = a(x) - \eta x^4,
\end{equation} which includes the ice-albedo effect. As temperatures rise, ice melts and the surface of the earth covered by ice decreases. The reduced reflecting surface yields an increased heat intake by oceans, which increases temperatures.

\begin{figure}[H]
    \centering
    \includegraphics[width = \linewidth]{../../../plots/albedo.png}
    \caption{The non-linearity of $a(x)$ given by the albedo effect. As temperature rises, more solar radiations are transferred to temperatures.}
\end{figure}

Equilibrium temperatures

\begin{figure}[H]
    \centering
    \includegraphics[width = \linewidth]{../../../plots/simulation-costante-e.png}
    \caption{Simulated path of temperatures, assuming no change in current emissions $e_0$, under the dynamics $\text{d}x$.}
    \label{fig:x-sim}
\end{figure}


\begin{figure}[H]
    \centering
    \includegraphics[width = \linewidth]{../../../plots/density_process.png}
    \caption{Conditional density of $x$, given a constant carbon concentration.}
    \label{fig:x-dens}
\end{figure}




\end{document}