\documentclass[../../main.tex]{subfiles}
\begin{document}

\subsection{Carbon concentration}

The carbon cycle is assumed to be comprised of two boxes: carbon in the atmosphere $m$ and carbon stored in natural sinks $m_s$. Carbon flows from the atmosphere to natural syncs at a rate $\delta_m(m_s)$, such that \begin{equation} \label{eq:dynamics:sink}
    \d m_s = \delta_m(m_s) \; m \; \d t.
\end{equation} The function $\delta_m$ is assumed to be decreasing, to model the reduced capacity of carbon sinks to store carbon.

Carbon concentration $m$ follows \begin{equation} \label{eq:dynamics:m:emissions}
    \d{m} = \Big(E - m \; \delta_m(m_s)\Big) \; \d{t} + \sigma^2_m \; m \; \d{w}.
\end{equation}  Following \citein{hambel_optimal_2021}, we rewrite the dynamics of carbon (\ref{eq:dynamics:m:emissions}) in terms of a business-as-usual growth rate $g$ and an abatement effort $\alpha$, by decomposing emissions in \begin{equation}
    E = m \Big( g + \delta_m(m_s) - \alpha \Big).
\end{equation} Doing this, allows us to rewrite (\ref{eq:dynamics:m:emissions}) as \begin{equation}\label{eq:dynamics:m:abatement}
    \frac{\d{m}}{m} = \d{\hat m} = \big(g - \alpha \big) \; \d{t} + \sigma^2_m \; \d w_m,
\end{equation} where $\hat m = \log m$.

Finally, letting $E_b$ be the business-as-usual emission rate, we let the emission control rate $\varepsilon = 1 - (E / E_b)$, be the percentage reduction of emissions compared to the business-as-usual scenario. As in \citein{hambel_optimal_2021}, we do not allow for negative emissions such that $\alpha \leq g + \delta_m(m_s)$.

\subsection{Temperature}

We employ a stylised Budyko–Ghil–Sellers energy balance model of global mean surface temperature (\cite{hogg_glacial_2008,ashwin_extreme_2020}). Radiative balance implies that incoming mean solar radiation $S_0$ and outgoing long-wave radiation $\eta x^4$ ought to balance. Yet, increasing greenhouse gasses in the atmosphere, compared to pre-industrial level $m_p$, generates additional forcing, $M_0 + M_1 \ln\left(m / m_p\right)$. In addition to these forces, to study the role of tipping points, we introduce the effects of land ice on the radiation balance, as done in (\cite{ghil_topics_2011,dijkstra_sensitivity_2015}). In particular, as temperatures rise, the area of ice caps, glaciers, and sea ice shrinks, which in turn reduces the planetary albedo $a(x)$, that is, the fraction of solar radiation reflected by earths' surface. We model this effect by letting the planetary albedo transition from a high level $a_1$ to a low level $a_2$ via a smooth transition function $\Sigma(x)$, which is equal to $0$ if $x < x_1$ and $1$ if $x > x_2$ for some temperature thresholds $x_1, x_2$\footnote{
    The function $\Sigma$ takes the form \begin{equation}
        \Sigma(x) = \frac{x - x_1}{x_2 - x_2} H(x - x_1) H(x_2 - x) + H(x - x_2),
    \end{equation} with $H(x) = \frac{1 + \tanh(x / x_a)}{2}.$
}. Under this specification, the albedo coefficient can be written as a function of temperature \begin{equation} \label{eq:assumption:albedo}
    a(x) = a_1 - \Big(1 - \Sigma(x)\Big) \; \underbrace{(a_1 - a_2)}_{\Delta a}.
\end{equation} 

\begin{figure}[H]
    \centering
    \input{../../../plots/albedo.tikz}
    \caption{Albedo coefficient $a(x)$ for different parametrisations of the long run coefficient $\alpha_2$}
    \label{fig:albedo_coefficient}
\end{figure}

It is the important to stress that the function $a$ (\ref{eq:assumption:albedo}) is a highly stylised average model for a complex and spatially heterogeneous process, which, in turn, puts a lot of uncertainty around the parameters $x_1, x_2, a_1,$ and $a_2$. In the robustness checks we explore the sensitivity of our results to these parameters.

Putting solar radiation, carbon concentration, and the albedo effect together we obtain the following temperature dynamics \begin{equation} \label{eq:dynamics:x}
    \epsilon \; \d x = \Big(\mu_x(x) + \mu_m(\hat m)\Big) \; \d t + \sigma^2_x \d w_x
\end{equation}

where $\epsilon$ is the oceans' heat capacity, \begin{equation}
    \mu_x(x) = S_0 \; \big(1 - a(x)\big) - \eta x^4,
\end{equation} and \begin{equation}
    \mu_m(\hat m) =  M_0 + M_1  \big(\hat m - \hat m_p \big).
\end{equation}

\subsection{Ice-albedo Feedback and Tipping Points}

The presence of the ice-albedo feedback effect $a(x)$ can introduce tipping points in the temperature dynamics. For higher losses in albedo due to melting of ice sheets, $\Delta a$, we obtain stronger feedbacks, which in turn can generate a regime in which, for a given carbon concentration level, there are two stable temperature regimes. This bifurcation can introduce jumps in the temperature process. To illustrate this, Figure (\ref{fig:nullclines}) depicts the equilibrium values of carbon concentration $x$ and temperature $m$, for four different potential albedo losses $\Delta a$. For low values of albedo loss (dark blue line), as the logarithm of carbon concentration increases, equilibrium temperatures rise linearly. For lower ice-free albedo levels (green line), an equivalent increase in carbon concentration yields larger increases in temperatures. As the loss in albedo rises, a bifurcation occurs and two stable equilibria of temperature arise.

\begin{figure}[H]
    \centering
    \input{../../../plots/nullcline.tikz}
    \caption{Nullclines of the dynamics $(x, m)$ for different parametrisations of albedo loss $\Delta a$.}
    \label{fig:nullclines}
\end{figure}

This bifurcation has strong implications for the dynamics of temperature. In Figure \ref{fig:bau:no-albedo}, I simulate the model assuming a business-as-usual scenario, that is, without abatement $\alpha = 0$ and assuming an emission growth consistent with the \textit{SSP5 - Baseline} scenario. Each marker represents a decade Despite its simplicity, the model presented here tracks the temperature and carbon concentration path in the baseline scenario very well.

\begin{figure}[H]
    \centering
    \input{../../../plots/baufig-noalbedo.tikz}
    \caption{Business as usual path of temperature with no albedo feedback loop, $\Delta a = 0$. Each marker represents the temperature at the start of each decade, starting from $2020$. In black, the \textit{SSP5 - Baseline} model.}
    \label{fig:bau:no-albedo}
\end{figure}

Under a larger albedo loss, $\Delta a = 0.08$, keeping the same emissions, the temperature exhibits a very different path. After passing a critical threshold, the temperature rises rapidly to the second steady state.

\textbf{
    TODO, 
    \begin{enumerate}
        \item Talk about the distribution and critical slowdown.
    \end{enumerate}
}

\begin{figure}[H]
    \centering
    \input{../../../plots/baufig.tikz}
    \caption{Business as usual path of temperature with positive albedo feedback, $\Delta a = 0.08$. Each marker represents the temperature at the start of each decade, starting from $2020$. In black, the \textit{SSP5 - Baseline} model.}
    \label{fig:bau:albedo}
\end{figure}

\begin{figure}[H]
    \centering
    \input{../../../plots/xdens.tikz}
    \caption{Density}
    \label{fig:bau:density}
\end{figure}




\end{document}