\documentclass[../../main.tex]{subfiles}
\begin{document}

The climate model employed here, and its calibration, is a variant of the Budyko–Sellers-type model developed by \citein{mendez_investigating_2021}, without carbon concentration delay in decay. The model consists in a coupled system of stochastic differential equations (SDEs) for atmospheric carbon concentration $m$, in p.p.m. of CO$_2$, and average world temperature $x$, in Celsius deviation from pre-industrial levels. Carbon concentration evolves as \begin{equation} \label{eq:climate-model-c}
    \text{d}m = (e - \delta m) \ \text{d} t,
\end{equation} where $e$ is the level of emissions and $\delta$ is a natural decay rate. Temperature evolves as \begin{equation} \label{eq:climate-model-x}
    \frac{1}{\kappa} \text{d}x = \Big( \mu(x) + S + M \log(m / m_{pre}) \Big) \text{d} t + \sigma^2 \text{d}w
\end{equation} where $m_{pre}$ is the pre-industrial level of carbon concentration, $M$ is the direct forcing of CO$_2$, and $w$ is a Wiener process. The non-linearity in temperature dynamics is given by the function \begin{equation}
    \mu(x) = a(x) - \eta x^4,
\end{equation} which models the ice-albedo effect. As temperatures rise, ice melts and the surface of the earth covered by ice decreases. The reduced reflecting surface yields an increased heat intake by oceans, which increases temperatures. Using the calibration by \citein{mendez_investigating_2021}, in Figure \ref{fig:albedo} I plot the baseline energy input\begin{equation}
    a(x) = a_1 (1 - \sigma(x)) + a_2 \ \sigma(x)
\end{equation} where $\sigma(x)$ is a smooth transition function\footnote{
    The transition function $\sigma$ has the form,
    \begin{equation}
        \sigma(x) = \frac{x - x_1}{x_2 - x_1} H(x - x_1) H(x_2 - x) + H(x - x_2)
    \end{equation}
    where $H = \frac{1 + \tanh(x / x_a)}{2} $
}. As temperature rises, the albedo coefficient of earth's surface transitions from $a_1$ to $a_2$ with $a_1 > a_2$. It is important to stress that the calibration in the current model is done to match current temperature and carbon concentration levels $(x_0, m_0)$, assuming that those are in equilibrium, hence there is still high uncertainty around the parameters.

\begin{figure}[H]
    \centering
    \includegraphics[width = 0.7\linewidth]{../../../plots/albedo.png}
    \caption{The non-linearity of $a(x)$ given by the albedo effect. As temperature rises, more solar radiations are transferred to temperatures.}
    \label{fig:albedo}
\end{figure}

To better understand the role of the tipping point in the dynamics, in Figure (\ref{fig:x-sim}) I show the temperature and CO$_2$ dynamics implied by the model if emissions are kept constant at the level of the average emissions of 2022, $e_0$. Each marker represents the value of temperature and carbon concentration every ten years. The dashed line is the curve along which temperature is in equilibrium given a certain level of carbon concentration. Interestingly, as the stock of CO$_2$ concentration $c$ rises, its relationship with temperature might appear linear for up to twenty years, before the ice-albedo feedback effect appears and the temperature rises quickly to its steady state level.


\begin{figure}[H]
    \centering
    \includegraphics[width = 0.7\linewidth]{../../../plots/sim-bse.png}
    \caption{Simulated path of temperatures, assuming no change in current emissions $e_0$, under the dynamics $\text{d}x$. Each marker represents the value of $(x, c)$ every ten years.}
    \label{fig:x-sim}
\end{figure}

The ice-albedo feedback affects not only the level of temperature, but also the distribution of temperature shocks. In the neighbourhood of the tipping point, there are longer lasting warmer periods. This phenomenon is known as critical slowing down. Compared to climate models with jump processes, the asymmetry in the distribution of temperature as the climate approaches the tipping point (see Figure \ref{fig:x-dens}) yields temperature damages that are on average larger, while also displaying increasing tipping probabilities.

\begin{figure}[H]
    \centering
    \includegraphics[width = \linewidth]{../../../plots/density_process.png}
    \caption{Conditional density of $x$, given a constant carbon concentration $c$.}
    \label{fig:x-dens}
\end{figure}




\end{document}