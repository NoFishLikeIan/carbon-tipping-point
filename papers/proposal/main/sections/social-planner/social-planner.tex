\documentclass[../../main.tex]{subfiles}
\begin{document}

To ease notation, in this section I will let the vector of state variables \begin{equation}
    u \coloneqq (x, m_s, \hat m, \hat y).
\end{equation}

To capture the degree of relative risk aversion of society $\gamma$, we assume a social planner with Epstein-Zin preferences. This is done to introduce relative risk aversion $\gamma$ and elasticity of intertemporal substitution $\psi$. Hence, given a level of consumption $c$ and of utility $v$ the preference function $f: \Re^2 \to \Re$ is given by \begin{equation}
    f(c, v) = \left( \left( \frac{c}{((1 - \gamma) v)^{\frac{1}{1 - \gamma}}} \right)^{1 - \frac{1}{\psi}} - 1 \right) v.
\end{equation}

The social planner's utility at time $t$ is \begin{equation} \label{eq:utility}
    \begin{split}
        v(u(t); \alpha, \chi) &= \E_t \int^{\infty}_t f\Big(c(u(s); \chi), v(u(s); \alpha, \chi) \Big) \d s.
    \end{split}
\end{equation} 

The social planner then seeks to maximise, for all $t$, its utility (\ref{eq:utility}), given an admissible abatement strategy $\alpha \in \mathcal{A}$ and consumption $\chi \in \mathcal{C}$\footnote{
    An admissible path of abatement is a function of time $\mathcal{A} = C([t, \infty), [0, g(t) + \delta_m(m^s)])$ and $\mathcal{C} = C([0, \infty), [0, 1])$.
}.

\begin{proposition}
    The value function satisfies the Hamilton-Bellman-Jacobi equation \begin{equation}
        \begin{split}
            0 = \sup_{\alpha, \chi} \; &f(c(u, \alpha), v(u; \alpha, \chi)) \\
            + \; &(\partial_x v) \; \mu(x, \hat m) + (\partial_{\hat m} v) \; \big(g(t) - \alpha \big) + (\partial_{m^s} v) \; \delta_m(m_s) \\
            + \; &(\partial_{\hat y} v) \; \big(\rho(t) + \phi(t, \chi) - b(t, u, \alpha) - d(x) \big) \\
            + \; &\sigma^2_m (\partial^2_{\hat m} v) + \sigma^2_x (\partial^2_x v)
        \end{split}
    \end{equation}
\end{proposition}

\end{document}