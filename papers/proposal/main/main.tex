\documentclass[american, abstract=off]{scrartcl}

    \newcommand{\lang}{en}

    \usepackage{babel}
    \usepackage[utf8]{inputenc}

    \usepackage{csquotes}

    \usepackage{amsmath, amssymb, mathtools, bbm, bm}
    \usepackage{xcolor}
    \usepackage{xcolor-solarized}
    \usepackage{graphicx}
    \usepackage{wrapfig}
    \usepackage{relsize}
    \usepackage{makecell}
    \usepackage{booktabs}
    \usepackage[font=footnotesize,labelfont=bf]{caption}
    \usepackage{subcaption}
    \usepackage{float}
    \usepackage{multirow} 

    % Refs
    \usepackage{hyperref}
    \usepackage{cleveref}
    \hypersetup{
        colorlinks = true, 
        urlcolor = blue,
        linkcolor = blue, 
        citecolor = blue 
      }      

    \usepackage{subfiles} % Load last

    % Paths

    % Formatting
    \setlength{\parindent}{0em}
    \setlength{\parskip}{0.5em}
    \setlength{\fboxsep}{1em}
    \newcommand\headercell[1]{\smash[b]{\begin{tabular}[t]{@{}c@{}} #1 \end{tabular}}}

    \newcommand{\note}[1]{
      \ifdraft{
        {\color{blue} \textbf{#1}}
      }{}
    }

    % Graphs
    \usepackage{tikz} 
    \usepackage{tikzit}
    \usepackage{pgfplots}
    \usepackage{ifdraft}
    \usetikzlibrary{positioning,fit,calc,decorations.pathreplacing,calligraphy, arrows.meta, pgfplots.groupplots}

    \pgfplotsset{compat=1.18}

    \newcommand{\inputtikz}[1]{%  

      \ifdraft{
        \includegraphics[width=0.5\textwidth]{example-image-a}
      }{
        \IfFileExists{#1}{
            \input{#1}
          }{
            \includegraphics[width=0.5\textwidth]{example-image-a}
        }
      }
    }

    % Math commands

    \DeclareMathOperator{\Dx}{\partial_x}
    \DeclareMathOperator{\Dc}{\partial_c}
    
    \let\d\relax
    \newcommand{\d}[1]{\mathrm{d}#1}

    \let\H\relax
    \DeclareMathOperator{\H}{\mathcal{H}}
    \let\Re\relax
    \DeclareMathOperator{\Re}{\mathbb{R}}

    \DeclareMathOperator{\E}{\mathbb{E}}
    \newcommand{\unit}[1]{\text{#1}}
    \newcommand{\control}[1]{\bm{#1}}

    \newtheorem{proposition}{Proposition}
    \newtheorem{definition}{Definition}


    % Bibliography

    \usepackage[bibencoding=utf8, style=apa, backend=biber, eprint=false]{biblatex}
    \addbibresource{../scc-tipping-points.bib}

    \newcommand{\citein}[1]{\citeauthor{#1} (\citeyear{#1})}
    
    \DeclareCaptionFormat{empty}{#1}

    % Make title page

    \author{Andrea Titton}
    \title{Climate Tipping Points and\\ Optimal Emissions}
    
\begin{document}

\maketitle

As the world temperature rises, due to carbon dioxide (CO2) emissions from human economic activities, the risk of tipping points in the climate system becomes more concrete (\cite{ashwin_extreme_2020,sledd_cloudier_2021}). This risk affects the social cost of carbon, the marginal damage of increasing carbon emissions. In this paper, I study the relationship between the risk of tipping and the social cost of carbon. To do so, I solve a social-planner integrated model with a stylised ice-albedo feedback in the climate dynamics (\cite{hogg_glacial_2008,ashwin_tipping_2012}). I model a tipping point induced by the ice–albedo feedback and study how this affects optimal abatements. The tipping point affects temperature dynamics, and as a consequence optimal emissions, in three ways. First, it introduces a non-linear increase in temperature. Second, it makes positive temperature shocks more persistent than negative ones. Third, it introduces a jump in the abatement necessary to revert temperatures to the pre-tipping-point level. I show that, in this context, it is crucial not only to quickly reach net-zero emissions, but to also flatten the emission curve to reduce the risk of tipping.

The importance of modelling precise climate dynamics and tipping points when determining optimal emission paths has been increasingly recognised in economics (\cite{van_den_bremer_risk-adjusted_2021,dietz_economic_2021,dietz_are_2020,taconet_social_2021,lontzek_stochastic_2015}). Previous approaches have mostly focused on the stochastic nature of tipping points, by modelling temperature dynamics or damages as jump processes, with arrival rates increasing in emissions. Yet, many tipping points in the climate system are caused by bifurcations (\cite{ashwin_extreme_2020,ashwin_tipping_2012}). In this paper, I show that introducing this class of tipping points in an integrated model yields similar prediction in terms of aggregate emissions, but prescribes much steeper reduction of emissions to keep the risk of tipping low.

To tease out this difference, I study an AK-model in which increases temperature beyond pre-industrial levels reduce economic growth (as in \cite{pindyck_economic_2013} and \cite{hambel_optimal_2021}). This modelling choice, as opposed to having temperatures wipe-out a fraction of the capital stock, as in Nordhaus (\citeyear{nordhaus_estimates_2014,nordhaus_question_2008,nordhaus_revisiting_2017}), is motivated by recent evidence on the role of temperature in reducing economic growth and productivity (\cite{burke_global_2015, dietz_growth_2019}). \note{Add punchline.}

\section{Notation}

To standardise notation between the climate model and the economic model, throughout the paper, I use capital letters to denote state variables, for example $X(t)$. In general, to avoid congestion the dependence on time is $t$ implied, for example \begin{equation}
  \d{X} = f(X) \; \d{t} + \sigma^2 \; \d{w}.
\end{equation} Both rates and log-transformed variables are denoted with lower case letters, $x \coloneqq \log X$. Parameters, whether they are time dependent or not, are denoted with lower case Greek letters. Finally, control variables are denoted by bold letters $\control{\alpha}$ or $\control{E}$.

\section{Climate Model}

\subfile{sections/climate-model/climate-model.tex}

\section{Economy}

\subfile{sections/economy-model/economy-model.tex}

\section{Social Planner Problem}

\subfile{sections/social-planner/social-planner.tex}

\section{Numerical Solution}

\subfile{sections/numerical-solution/numerical-solution.tex}

\ifdraft{}{
  \nocite{*}
  \newpage
  \printbibliography
}

\end{document}