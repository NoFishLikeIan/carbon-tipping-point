\documentclass[american, abstract=on]{scrartcl}

    \newcommand{\lang}{en}

    \usepackage{babel}
    \usepackage[utf8]{inputenc}

    \usepackage{csquotes}

    \usepackage{amsmath, amssymb, mathtools, bbm, bm}
    \usepackage{xcolor}
    \usepackage{xcolor-solarized}
    \usepackage{graphicx}
    \usepackage{wrapfig}
    \usepackage{relsize}
    \usepackage{makecell}
    \usepackage{booktabs}
    \usepackage[font=footnotesize,labelfont=bf]{caption}
    \usepackage{subcaption}
    \usepackage{float}
    \usepackage{multirow} 

    % Refs
    \usepackage{hyperref}
    \usepackage{cleveref}
    \hypersetup{
        colorlinks = true, 
        urlcolor = blue,
        linkcolor = blue, 
        citecolor = blue 
      }      

    \usepackage{subfiles} % Load last

    % Paths

    % Formatting
    \setlength{\parindent}{0em}
    \setlength{\parskip}{0.5em}
    \setlength{\fboxsep}{1em}
    \newcommand\headercell[1]{\smash[b]{\begin{tabular}[t]{@{}c@{}} #1 \end{tabular}}}

    % Graphs
    \usepackage{tikz} 
    \usepackage{tikzit}
    \usetikzlibrary{positioning,fit,calc,decorations.pathreplacing,calligraphy, arrows.meta}

    % Math commands

    % Bibliography

    \usepackage[bibencoding=utf8, style=apa, backend=biber, eprint=false]{biblatex}
    \addbibresource{../scc-tipping-points.bib}

    \newcommand{\citein}[1]{\citeauthor{#1} (\citeyear{#1})}    

    % Make title page

    \author{Andrea Titton}
    \title{Climate Tipping Points and\\the Social Cost of Carbon}
    
\begin{document}

\maketitle

As the world temperature rises, due to carbon dioxide (CO2) emissions from human economic activities, the risk of tipping points in the climate system becomes more concrete. This risk affects the social cost of carbon (SCC), that is, the marginal economic damage of a increasing carbon emissions. In this paper, I study the relationship between the risk of tipping and the SCC. To do so, I solve a simple planner economy interacting with a with a climate model
with glacial-interglacial transitions (\cite{sellers_global_1969,mcgehee_quadratic_2014}). 

% TODO: Talk more about climate tipping points

Tipping points in temperature affect the SCC in three ways. First, they introduce a discontinuous jump in economic damages as temperature increases. Second, they make it more costly to revert temperature back to pre-tipping level. Third, they make warm shocks in temperature more persistent, also known as critical slowdown. Previous approaches in modelling tipping points have been to introduce Poisson jumps, with emission-dependant probabilities, in the temperature process. Yet, these cannot accommodate for the latter two effects of climate tipping points (cite). In this paper, I address this problem.


\subfile{sections/introduction/introduction.tex}

\section{Related literature}
\section{Climate Model}

\subfile{sections/climate-model/climate-model.tex}

\section{Economy}

\subfile{sections/economy-model/economy-model.tex}

\section{Model without Tipping Point}

\subfile{sections/linear-model/linear-model.tex}

\section{Model with Tipping Points}

\subsection{Deterministic skeleton}
\subfile{sections/deterministic-skeleton/deterministic-skeleton.tex}

\nocite{*}
\newpage
\printbibliography


\end{document}