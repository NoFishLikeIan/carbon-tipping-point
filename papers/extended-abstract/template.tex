\documentclass[12pt]{article}
\usepackage[paper=letterpaper,margin=2cm]{geometry}
\usepackage{amsmath}
\usepackage{natbib}
\usepackage{graphicx}
\usepackage{hyperref}
\usepackage{caption} 

\newcommand{\citein}[1]{\citeauthor{#1}, \citeyear{#1}}

% Graphs
\usepackage{float}
\usepackage{tikz} 
\usepackage{tikzit}
\usepackage{pgfplots}
\usepackage{ifdraft}
\usetikzlibrary{positioning,fit,calc,decorations.pathreplacing,calligraphy, arrows.meta, pgfplots.groupplots}

\pgfplotsset{compat=1.18}

\newcommand{\includegraphics[width = 0.5\linewidth]}[1]{%  
    \ifdraft{
    \includegraphics[width=0.5\textwidth]{example-image-a}
    }{
    \IfFileExists{#1}{
        \input{#1}
        }{
        \includegraphics[width=0.5\textwidth]{example-image-a}
    }
    }
}

\title{Climate Tipping Points and Optimal Emissions}
\author{} %Leave this empty, document must be anonymous
\date{} %Leave this empty

\begin{document}

\maketitle

\textbf{Keywords:} Social Cost of Carbon, Tipping Points, Climate Change, Environmental Economics

\section{Introduction}

As the world temperature rises, due to carbon dioxide (CO$_2$) emissions from human economic activities, the risk of tipping points in the climate system becomes more concrete \citep{ashwin_extreme_2020,sledd_cloudier_2021}. This risk affects the social cost of carbon, that is, the marginal damage of increasing carbon emissions. In this paper, I study the relationship between the risk of tipping, optimal emissions, and the social cost of carbon. To do so, I solve a social-planner integrated model \citep{hambel_optimal_2021}, with a stylised ice-albedo feedback in the climate dynamics \citep{hogg_glacial_2008,ashwin_tipping_2012}. I model a tipping point induced by the ice–albedo feedback and study how this affects optimal abatement. The tipping point affects temperature dynamics, and as a consequence optimal emissions, in three ways. First, it introduces a non-linear increase in temperature. Second, it makes positive temperature shocks more persistent than negative ones. Third, it introduces a jump in the abatement necessary to revert temperatures to the pre-tipping-point level. I show that, in this context, it is crucial not only to quickly reach net-zero emissions, but to also flatten the emission curve to reduce the risk of tipping.

\section{Methodology}

I study the social planner solution to an integrated model with non-linear climate dynamics. The non-linearity is due to the presence of an ice-albedo feedback. A planet's albedo is the fraction of solar radiation that is reflected by its surface and, hence, does not contribute to warming. Since ice is highly reflective, Earth's ice coverage contributes significantly to increasing its albedo. As temperature rises, the Earth's surface area covered in ice recedes, which reduces the albedo, which, in turns, exacerbates heating. I model this by assuming that the fraction of reflected incoming solar radiation is a decreasing function of temperature, as displayed in Figure \ref{fig:albedo} \citep{oerlemans_ice_1984, mendez_investigating_2021}. The modelled ice dynamics are highly stylised to keep the model tractable while, at the same time, introduce a relevant bifurcation tipping point in the climate dynamics. Throughout the paper, optimal emissions, investment, and consumption paths are computed vis-à-vis a business-as-usual scenario, which is calibrated using the SSP5 IPCC scenario \citep{kriegler_fossil-fueled_2017}.

\begin{figure}[H]
    \centering
    \resizebox{0.6\linewidth}{!}{\includegraphics[width = 0.5\linewidth]{../../plots/albedo.tikz}}
    \caption{Albedo coefficients at different levels of temperature deviations from pre-industrial levels $T^{\mathrm{p}}$ and different parametrisations.}
    \label{fig:albedo}
\end{figure}

The climate affects the economy as increases in temperature beyond pre-industrial levels reduce economic growth \citep{pindyck_economic_2013, hambel_optimal_2021}. This modelling choice, as opposed to having temperatures wipe-out a fraction of the capital stock, as in Nordhaus (\citeyear{nordhaus_question_2008, nordhaus_estimates_2014,nordhaus_revisiting_2017}), is motivated by recent evidence on the role of temperature in reducing economic growth and productivity \citep{burke_global_2015, dietz_growth_2019}. To introduce this channel, I set up a simple Ramsey growth economy in which temperature deviations increase the rate of capital depreciation \citep{hambel_optimal_2021}. Production emits CO$_2$, yet, the level of emission associated with one unit of output can be abated by investing in abatement technologies, at increasing marginal costs. Hence, when computing optimal emission paths, the social planner attempts to balance the costs of substituting capital with that of climate damages. Finally, a link between the tipping probabilities and the social cost of carbon is established analytically using perturbation techniques developed by \cite{grass_small-noise_2015}.

\section{Results}

The model has been calibrated using the SSP5 projections from the IPCC \citep{kriegler_fossil-fueled_2017} and the optimal abatement and emission paths have been computationally computed, for different parametrisations of the climate dynamics. To understand the role of the ice-albedo feedback in driving temperature dynamics, in Figure \ref{fig:bau} I show the path of emissions and carbon concentration under the business-as-usual emission. The left and right panel assume a low and high albedo loss, respectively. The dashed line represents the equilibrium values of carbon concentration and temperature. On the one hand, if the albedo loss is low, equilibrium temperature rises linearly in the logarithm of carbon concentration. In this case the optimal emission path is consistent with the one found in \cite{nordhaus_revisiting_2017} and \cite{hambel_optimal_2021}. On the other hand, if albedo loss is high, there are two levels of temperature which are in equilibrium with a given level of carbon concentration, which introduces a tipping point into the system. In this case, as carbon concentration rises, positive temperature shocks become more persistent, until a tipping point is reached, after which temperature rises rapidly to a new steady state level. Firstly, this yields more ambitious optimal abatement and, hence, higher optimal social cost of carbon, compared to the benchmark models \citep{nordhaus_revisiting_2017,hambel_optimal_2021}. Secondly, this yields similar long run abatement levels to that prescribed by models with climate disaster events \citep{van_den_bremer_risk-adjusted_2021,dietz_economic_2021, lin_social_2023}, but, unlike these models, it prescribes larger social cost of carbon. This occurs because, in the presence of bifurcation tipping points, what matters are not only cumulative aggregate emissions but also emission paths. If abatement is delayed, the climate system is brought closer to the tipping point, which increases the probability of tipping, and, in turn, increases the social cost of carbon. Under this setting, optimal abatement is not only more ambitious in scope, but also in timing. 


\begin{figure}[H]
    \centering
    \resizebox{0.7\linewidth}{!}{\includegraphics[width = 0.5\linewidth]{../../plots/baufig.tikz}}
    \caption{Business as usual path of temperature with small (left) and large (right) albedo drop. Each marker represents the temperature every 20 years, starting from 2020. In black, the \textit{SSP5 - Baseline} model.}
    \label{fig:bau}
\end{figure}


I am in the progress of deriving an analytical link between the probability of tipping, and the social cost of carbon \citep{grass_small-noise_2015}. In addition, I plan to implement robustness checks for the  simplified albedo dynamics against more realistic climate models containing ice-albedo feedback \citep{seaver_wang_mechanisms_2023}.

\section{Conclusions}


The importance of modelling precise climate dynamics and tipping points when determining optimal emission paths has been increasingly recognised in economics \citep{lontzek_stochastic_2015,dietz_are_2020, van_den_bremer_risk-adjusted_2021,dietz_economic_2021,taconet_social_2021}. Previous approaches have mostly focused on the stochastic nature of tipping points, by modelling temperature dynamics or damages as jump processes, with arrival rates increasing in emissions. Yet, many tipping points in the climate system are caused by bifurcations \citep{ashwin_tipping_2012, ashwin_extreme_2020}. I show that introducing this class of tipping points into an integrated model yields similar prediction in terms of aggregate emissions, but prescribes much steeper reduction of emissions to keep the risk of tipping low. To the best of my knowledge, this paper is the first to integrate non-linear dynamics in assessing optimal emissions and the social cost of carbon.

\newpage
\bibliographystyle{apalike}
\bibliography{../scc-tipping-points.bib}

\end{document}
