\documentclass[../../main.tex]{subfiles}
\begin{document}

\subsection{Approximating Markov Chain}

Th numerical method employed here adapts that presented in \citein{kushner_numerical_2001}. First, we define a suitably large domain for the state variables $\mathcal{X} \subseteq \mathbb{R}^3$ and let $x = (T, m, y) \in \mathcal{X}$ be the state vector. Then, let $u = (\chi, \alpha) \in \mathcal{U} \coloneqq [0, 1] \times [0, \gamma^\b]$ be the vector of controls. Then we can define the operator \begin{equation}
    \begin{split}
        \mathcal{L}_t^u = \; &\frac{\mu(T, m)}{\epsilon} \frac{\partial}{\partial T} + \big(\varrho + \phi(\chi) - \delta_k - d(T) - A \beta(\alpha) \big) \frac{\partial}{\partial y} + \\ 
        &(\gamma^\b - \alpha)\frac{\partial}{\partial m} + \frac{(\sigma_T / \epsilon)^2}{2} \frac{\partial^2}{\partial T^2} + \frac{\sigma^2_k}{2} \frac{\partial^2}{\partial y^2} 
    \end{split}
\end{equation} such that the value functional at time $t$ satisfies \begin{equation} \label{eq:operator-definition}
    -\partial_t V_t =  \sup_{u} \; \mathcal{L}_t^u V_t + f(\chi, y, V_t).
\end{equation}

We seek to define a Markov chain consistent with (\ref{eq:operator-definition}), over a finite grid in the unit cube \begin{equation}\Omega_h = \{0, h, 2h \ldots 1 - h, 1\}^3.\end{equation} First we define the state dynamics over the unit cube by letting $\tilde{X} = X / \lvert\mathcal{X}\rvert$ where \begin{equation}
    \mathcal{X} = [T^p, T^p + \Delta T] \times [m_0, \overline{m}] \times [y_0, \overline{y}]
\end{equation} and defining the dynamics \begin{equation}
    d \tilde{X} = \omega(t, X, u) \; \d{t} + \Sigma \; \d w
\end{equation} where \begin{equation}
    \omega(t, X, u) = \begin{pmatrix}
        \mu(T, m) / \epsilon \Delta T \\
        (\gamma^{\b} - \alpha) / (\overline{m} - m_0) \\
        \big(\varrho + \phi(\chi) - \delta_k - d(T) - A \beta(\alpha) \big) / (\overline{y} - y_0)
    \end{pmatrix} 
\end{equation} and \begin{equation}
    \Sigma = \begin{pmatrix}
        \sigma_T / \epsilon \Delta T  & 0 & 0 \\
        0 & 0 & 0 \\
        0 & 0 & \sigma_K / (\overline{y} - y_0)\\
    \end{pmatrix}.
\end{equation}

For a given state $X_i$ we can now define the transition probabilities. Let \begin{equation}
    Q(X_i) = \left(\frac{\sigma_T}{\epsilon \Delta T}\right)^2 + \left(\frac{\sigma_K}{\overline{y} - y_0}\right)^2 + h \max_u \; \lvert \omega(t, X_i, u)  \rvert
\end{equation} then \begin{equation}
    p(X_i, X_i \pm h \Delta T) = \frac{\frac{\sigma^2_T}{2(\epsilon \Delta T)^2} + h \; \omega^{\pm}_T(t, X_i, u)}{Q(X_i)}
\end{equation}

\subsection{Post-transition phase}

We assume that at some point in the future $\tau \gg 0$, the abatement rate is equal to emission growth, $\gamma^b = \alpha$, and technological progress caps, $\varrho = 0$, such that the state variables evolve according to dynamics \begin{align}
    \d{m} &= 0, \\
    \epsilon \; \d T &= \mu(T, m) \; \d t + \sigma_T \; \d w_1 \text{ and } \\
    \d y &= \big( \phi(\chi) - \delta_k^p - d(T) \big) \; \d t + \sigma_k \; d w_2.
\end{align} I call this the \textit{post-transition} phase. We can then compute a steady state value function $V_t \eqqcolon \overline{V}$ for all $t \geq \tau$, which satisfies the Hamilton-Bellman-Jacobi equation \begin{equation}
    0 = \overline{\mathcal{L}}^\chi \; \overline{V} + f(\chi, y, \overline{V}),
\end{equation} where \begin{equation}
    \begin{split}
        \overline{\mathcal{L}}^\chi = \; &\frac{\mu(T, m)}{\epsilon} \frac{\partial}{\partial T} + \Big(\phi(\chi) - \delta_k - d(T) \Big) \frac{\partial}{\partial y} + \frac{(\sigma_T / \epsilon)^2}{2} \frac{\partial^2}{\partial T^2} + \frac{\sigma^2_k}{2} \frac{\partial^2}{\partial y^2}.
    \end{split}
\end{equation}

\end{document}