\documentclass[../../main.tex]{subfiles}
\begin{document}

\subsection{Markov Chain approximation}

Let the state vector be $X = (T, m, y)$ and the control vector $u = (\chi, \alpha) \in \mathcal{U} = [0, 1] \times [0, \gamma^\b]$. Then we can define the operator \begin{equation}
    \begin{split}
        \mathcal{L}_t^u = \; &\frac{\mu(T, m)}{\epsilon} \frac{\partial}{\partial T} + \big(\varrho + \phi(\chi) - \delta_k - d(T) - A \beta(\alpha) \big) \frac{\partial}{\partial y} + (\gamma^\b - \alpha)\frac{\partial}{\partial m} + \\
        &\frac{(\sigma_T / \epsilon)^2}{2} \frac{\partial^2}{\partial T^2} + \frac{\sigma^2_y}{2} \frac{\partial^2}{\partial y^2} 
    \end{split}
\end{equation} the objective functional at time $t$ satisfies \begin{equation} \label{eq:operator-definition}
    -\partial_t J_t =  \mathcal{L}_t^u J_t + f(\chi, y, J_t).
\end{equation}

We seek to define a Markov chain consistent with (\ref{eq:operator-definition}), over the unit cube \begin{equation}\Omega = \{0, h, 2h \ldots 1 - h, 1\}^3.\end{equation} First we define the state dynamics over the unit cube by letting $\tilde{X} = X / \lvert\mathcal{X}\rvert$ where \begin{equation}
    \mathcal{X} = [T^p, T^p + \Delta T] \times [m_0, \overline{m}] \times [y_0, \overline{y}]
\end{equation} and defining the dynamics \begin{equation}
    d \tilde{X} = \omega(t, X, u) \; \d{t} + \Sigma \; \d w
\end{equation} where \begin{equation}
    \omega(t, X, u) = \begin{pmatrix}
        \mu(T, m) / \epsilon \Delta T \\
        (\gamma^{\b} - \alpha) / (\overline{m} - m_0) \\
        \big(\varrho + \phi(\chi) - \delta_k - d(T) - A \beta(\alpha) \big) / (\overline{y} - y_0)
    \end{pmatrix} 
\end{equation} and \begin{equation}
    \Sigma = \begin{pmatrix}
        \sigma_T / \epsilon \Delta T  & 0 & 0 \\
        0 & 0 & 0 \\
        0 & 0 & \sigma_K / (\overline{y} - y_0)\\
    \end{pmatrix}.
\end{equation}

For a given state $X_i$ we can now define the transition probabilities. Let \begin{equation}
    Q(X_i) = \left(\frac{\sigma_T}{\epsilon \Delta T}\right)^2 + \left(\frac{\sigma_K}{\overline{y} - y_0}\right)^2 + h \max_u \; \lvert \omega(t, X_i, u)  \rvert
\end{equation} then \begin{equation}
    p(X_i, X_i \pm h \Delta T) = \frac{\frac{\sigma^2_T}{2(\epsilon \Delta T)^2} + h \; \omega^{\pm}_T(t, X_i, u)}{Q(X_i)}
\end{equation}

\subsection{Post-transition phase}

We suppose that at some point in the future $\tau$, abatement is free and is exogenously imposed such that carbon concentration remains constant, $\gamma^b = \alpha$. I call this the \textit{post-transition} phase, which the following climate dynamics \begin{align}
    \d{m} &= 0, \\
    \epsilon \; \d T &= \mu(T, m) \; \d t + \sigma_T \; \d W \text{ and } \\
    \d y &= \big( \varrho + \phi(\chi) - \delta_k^p - d(T) \big) \; \d t.
\end{align}

Let $p(T, \overline{m})$ be the steady state density of temperature $T$ given the constant level of carbon concentration $\bar{m}$. It can be shown that \begin{equation}
    p(T, \overline{m}) \propto \exp\left(-\frac{\mu_m(\overline{m}) T + (1 - \lambda_1) S_0 T - \frac{\eta}{5} T^5 + S_0 (\lambda_1 - \lambda_2) \log(1 + \exp(T - T_i))}{\sigma_T^2 / 2\epsilon}\right).
\end{equation}

Ito's lemma on $d(T)$.
\begin{equation}
    \epsilon \; \d d(T) = \left(\mu(T, \overline{m}) d'(T) + \frac{\sigma^2_T}{2\epsilon} d''(T) \right) \d t + \sigma_T d'(T) \d W.
\end{equation}

\newpage

Let $\bar{V}(t, T, m, y)$ be the post-transition value function, which satisfies the Hamilton-Bellman-Jacobi equation \begin{equation}
    0 = \sup_{\chi} \Big\{f(\chi, y, \bar{V}) +  \partial_y \bar{V} \phi(\chi) \Big\} - \partial_y \bar{V} \Big(\delta^{p}_k + d(T)\Big) + \partial_T \bar{V} \; \frac{\mu(T, m)}{\epsilon}+ \partial^2_T \bar{V} \; \frac{\sigma^2_T}{2\epsilon}.
\end{equation}

\end{document}