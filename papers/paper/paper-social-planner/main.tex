\documentclass[11pt]{article}
    \usepackage{style}
    \newcommand{\lang}{en}
    
    \usepgfplotslibrary{external} 
    \tikzexternalize

    \usetikzlibrary{positioning,fit,calc,decorations.pathreplacing,calligraphy, arrows.meta, pgfplots.groupplots}   

    \pgfplotsset{compat=1.18}

    \newcommand{\inputtikz}[1]{%  
        \IfFileExists{#1}{\input{#1}}{
            \includegraphics[width=0.5\textwidth]{example-image-a}
        }
    }

    % Math commands

    \DeclareMathOperator{\Dx}{\partial_x}
    \DeclareMathOperator{\Dc}{\partial_c}
    
    \let\d\relax
    \newcommand{\d}[1]{\mathrm{d}#1}

    \let\H\relax
    \DeclareMathOperator{\H}{\mathcal{H}}
    \let\Re\relax
    \DeclareMathOperator{\Re}{\mathbb{R}}

    \DeclareMathOperator{\E}{\mathbb{E}}
    \newcommand{\unit}[1]{\text{#1}}
    \newcommand{\control}[1]{#1}
    \renewcommand{\b}{\mathrm{b}}


    \usepackage{subfiles} % Load last
    

    % Make title page

    \author{Andrea Titton\thanks{
      Faculty of Economics and Business, University of Amsterdam (a.titton@uva.nl). I thank my supervisors Florian Wagener and Cees Diks for the patient guidance on this paper. I also thank the CeNDEF group and the Quantitative Economics section at the University of Amsterdam for helpful comments throughout the many seminars. Finally, I thank the attendees of the SEEMS/ASF seminar series for their feedback.
    }}
    \title{The Dangers of Abatement Procrastination in a Tipping Climate}

\begin{document}

\maketitle

\begin{abstract}
  This paper investigates the role of non-linear tipping points in determining optimal abatement policies. To do so, I introduce a stylised ice-albedo tipping point in the climate dynamics and study the consequences this has in determining optimal emissions in a dynamic stochastic general equilibrium model. In line with recent evidence, I assume that climate change hinders economic growth. I show that the presence of a tipping point prescribes ambitious abatement policies, not only in scope but, crucially, in timing.
\end{abstract}

\newpage
\subfile{sections/introduction/introduction.tex}

\section{Climate Model}

\subfile{sections/climate-model/climate-model.tex}

\section{Economy}

\subfile{sections/economy-model/economy-model.tex}


\section{Social Planner Problem}

\subfile{sections/social-planner/social-planner.tex}

\section{Benchmark model: Stochastic Tipping}

\subfile{sections/benchmark/benchmark.tex}

\section{Main Results}

\subfile{sections/optimal-emissions/optimal-emissions.tex}

\section{Conclusion}

This paper studies the role of tipping points in determining optimal emissions. Building on the calibration by \cite{hambel_optimal_2021}, I extend the climate dynamics to include a potential bifurcation induced by the loss in albedo due to the change in the area of ice caps, sea ice, and glaciers \cite{ashwin_tipping_2012,ashwin_extreme_2020}. I show that, in the presence of tipping points, optimal abatement is more ambitious in scope and timing. In fact, early abatement is crucial to avoid long periods of exposure to tipping risk.

The model presented here represents an early and simplified analysis that can be extended in various directions. First, more work is needed to analytically link the risk of tipping and the optimal abatement strategy, in order to quantify precisely the role of higher order climate dynamics in determining the social cost of carbon. Second, the underlying assumption of the social planner's optimisation problem is that she knows the climate dynamics and the role of the ice-albedo feedback. Such an assumption calls for extending the analysis to a situation in which the magnitude of the albedo loss is not known and rather can be estimated using early warning signals. Yet, in the face of uncertainty, the optimal abatement policy derived in this paper serves as a good rule against the possible worst case scenario.

\newpage
\bibliography{scc-tipping-points}

\newpage\appendix
\subfile{sections/numerical-solution/numerical-solution.tex}

\end{document}