\documentclass[../../main.tex]{subfiles}
\begin{document}

As the world temperature rises, due to carbon dioxide (CO2) emissions from human economic activities, the risk of tipping points in the climate system becomes more concrete \citep{ashwin_extreme_2020,sledd_cloudier_2021}. This risk affects the trade-off between the economic gains from emissions and the damages such emissions impose on the economy. In this paper, I study the relationship between the presence of a tipping point and the optimal abatement of emissions. To do so, I solve a social-planner integrated assessment model with a stylised ice-albedo feedback in the climate dynamics \citep{hogg_glacial_2008,ashwin_tipping_2012} and study the effect this has on optimal abatement policies. The tipping point affects temperature dynamics, and as a consequence optimal emissions, in three ways. First, it introduces a non-linear increase in temperature. Second, it makes positive temperature shocks more persistent than negative ones. Third, it introduces a jump in the abatement necessary to revert temperatures to the pre-tipping-point level. I show that, when tipping points are present, optimal abatement is more ambitious in scope and timing

The importance of modelling precise climate dynamics and tipping points when determining optimal emission paths has been increasingly recognised in economics \citep{van_den_bremer_risk-adjusted_2021,dietz_economic_2021,dietz_are_2020,taconet_social_2021,lontzek_stochastic_2015}. Previous approaches have mostly focused on the stochastic nature of tipping points, by modelling temperature dynamics \citep{dietz_economic_2021} or damages \citep{lontzek_stochastic_2015} as jump processes, with arrival rates increasing in emissions. Yet, many tipping points in the climate system are caused by bifurcations \citep{ashwin_extreme_2020,ashwin_tipping_2012}. In this paper, I show that introducing this class of tipping points in an integrated assessment model yields similar predictions in terms of aggregate emissions, but prescribes much steeper reduction of emissions to keep the risk of tipping low.

To tease out this difference, I study an AK-model in which increases in temperature, beyond pre-industrial levels, reduce economic growth (as in \cite{pindyck_economic_2013} and \cite{hambel_optimal_2021}). This modelling choice, as opposed to having temperatures wipe-out a fraction of the capital stock, as in \citeauthor{nordhaus_estimates_2014} (\citeyear{nordhaus_estimates_2014,nordhaus_revisiting_2017}), is motivated by recent evidence on the role of temperature in reducing economic growth and productivity \citep{burke_global_2015, dietz_growth_2019}.

\end{document}