\documentclass[../../main.tex]{subfiles}
\begin{document}

Before analysing optimal emission with tipping points, this section introduces a benchmark model with stochastic tipping. Hereafter, I refer to the model as \textit{Stochastic Tipping model}. The Stochastic Tipping model is a widely used in the economic literature to approximate tipping points in the climate dynamics (e.g. \citealt{hambel_optimal_2021}). Comparing the model developed in this paper with the Stochastic Tipping model allows us to determine if and how the optimal abatement differ and, as a consequence, what the approximation misses.

To establish a meaningful benchmark, I will assume that the albedo is constant in temperature, namely $\lambda(T) = \lambda_1$, such that the contribution of temperature to forcing (\ref{eq:forcing:temperature}) is given by \begin{equation}
    \mu^{\mathrm{ST}}_T(T) \coloneqq S_0 (1 - \mathbf{\lambda_1}) - \eta\sigma T^4.
\end{equation} The tipping point does not arise due to the albedo coefficient changing, but is modelled as a counting process $N$ with arrival rate $\pi(T)$ and intensity $\Theta(T)$, both increasing in temperature. Intuitively, as temperature rises, the risk of tipping $\pi(T)$ and the size of the temperature increase $\Theta(T)$ grow. Then temperature dynamics in the Stochastic Tipping model follow \begin{equation}
    \epsilon \; \dd{T} = \big(\mu^{\mathrm{ST}}_T(T)  + \mu_m(m) \big) \; \dd{t} + \sigma_x \dd{w}^{\mathrm{ST}} + \Theta(T) \dd{N}.
\end{equation} Following \cite{hambel_optimal_2021}, the calibrated arrival rate and temperature increase are calibrated as \begin{align}
    \pi(T) &=  -\frac{1}{4} + \frac{0.95}{1 + 2.8 e^{-0.3325 (T - T^{\mathrm{P}})}} \text{ and } \\
    \Theta(T) &= -0.0577 + 0.0568 (T - T^{\mathrm{P}})-0.0029(T - T^{\mathrm{P}})^2.
\end{align}

\end{document}