\documentclass[../../main.tex]{subfiles}
\begin{document}

To derive the optimal abatement path and the social cost of carbon, I solve the problem of a social planner who derives utility from consumption, is risk averse, and discounts the future. To disentangle the role of these two components I model the social planner as having Epstein-Zin preferences. Societal utility $U$ at time $t$, given an abatement strategy $\control{\alpha}$ and a consumption schedule $\control{\chi}$, is defined recursively by the integral equation as \begin{equation} \label{eq:utility}
    U(t; \control{\alpha}(t), \control{\chi}(t)) = \E_t \int^{\infty}_t f\left(C(s), U(s; \control{\alpha}(s), \control{\chi}(s))\right) \; \d{s}
\end{equation} where $C(s) = \control{\chi}(s) Y(s)$ is the consumption path. Introducing the coefficient of relative risk aversion $\theta > 1$, elasticity of intertemporal substitution $\psi > 0$, and the discount rate $\rho$, the Epstein-Zin aggregator \citep{duffie_asset_1992} is defined as \begin{equation} \label{eq:aggregator}
    f(C, U) = \frac{\rho}{1 - 1 / \psi} (1 - \theta) U  \left( \left(\frac{C}{((1 - \theta) U)^{\frac{1}{1 - \theta}}}\right)^{1 - 1 / \psi} - 1 \right).
\end{equation} Given the optimal abatement and consumption schedule $\control{\alpha}$ and $\control{\chi}$, let the value function be defined recursively as \begin{equation}
    V_t(T, m, y) = \sup_{\control{\chi}, \control{\alpha}} \E_t \int^{\infty}_t f(C, V_s) \; \d{s},
\end{equation} which satisfies the  Hamilton-Jacobi-Bellman (HJB) \begin{equation}
    \begin{split}
        -\partial_t V_t = \; &f(C, V_t) + \partial_T V_t \; \mu(T, m) + \partial_m V_t \; (\gamma_t - \alpha) \; + \\
        &\partial_y V_t \; \Big(\varrho + \phi(\chi) - \delta_k - A \beta(\varepsilon) - d(T) \Big) \; + \\
        &\frac{\sigma^2_k}{2} \partial^2_y V_t + \frac{\sigma^2_T}{2} \partial^2_T V_t
    \end{split}
\end{equation}

The HJB equation is then solved numerically to obtain the value function at time $t = 0$, $V_0$, and obtain optimal consumption and abatement schedules. The details of the numerical procedure are laid out in Appendix \ref{appendix:assumptions} and \ref{appendix:approximating-markov-chain}.


\end{document}