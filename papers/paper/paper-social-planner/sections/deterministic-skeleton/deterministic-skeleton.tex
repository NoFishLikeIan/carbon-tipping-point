\documentclass[../../main.tex]{subfiles}
\begin{document}

\renewcommand*{\arraystretch}{1.5}

We now go back to the general model. We know have two states $(x, c)$ and a control $e$. Letting $\lambda$ be the shadow price of emission, we deriva a system of differential equations for $u = (x, c, \lambda, e)'$. Starting from the Hamiltonian \begin{equation}
    \begin{split}
        H(x, c, \lambda, p_c) = \max_e &(\beta_0 - \tau + p_c) e - \frac{1}{2} \beta_1 e^2 + \frac{1}{2} \gamma (x - x_p)^2 \\
        - \ &p_c \delta c + \kappa \lambda_x(a(x) - \eta x^4 + S + A \log(c / c_p))
    \end{split}
\end{equation}

We obtain the system

\begin{equation}
    F(x, c, \lambda, e) = \begin{pmatrix}
        \kappa (a(x) - \eta x^4 + S + A \log(c / c_p)) \\
        e - \delta c \\
        (\rho - \kappa a'(x) + 4 \kappa \eta x^3) \lambda + \gamma (x - x_p) \\
        (\rho + \delta) (e - e_u) - \frac{\kappa A}{\beta_1} \frac{\lambda}{c}
    \end{pmatrix}
\end{equation}

\subsubsection{Nullcline}

If we seek to find the equilibria of the system defined by $F$ we must first compute nullclines. First, we start with the ``trivial'' nullcline, and set $c = e / \delta$. Let $E$, $X$, and $\Lambda$ be the space of emissions, temperature, and shadow price respectively, then we can find nullclines by solving the following diagram

\begin{center}
    \begin{tikzpicture}[node distance=2.2cm, auto, thick, >=Stealth]
    % Nodes
        \coordinate (A) at (-60:1.5);
        \coordinate (B) at (60:1.5);
        \coordinate (C) at (180:1.5);

        \node[draw, circle, fill=blue!20] (X) at (A) {$X$};
        \node[draw, circle, fill=green!20] (E) at (B) {$E$};
        \node[draw, circle, fill=red!20] (L) at (C) {$\Lambda$};
        % Edges
        \draw[->] (X) -- node[midway, right] {$\psi$} (E);
        \draw[->] (X) -- node[midway, below left] {$\omega$} (L);
        \draw[->] (E) -- node[midway, above] {$\varphi$} (L);
    \end{tikzpicture}
\end{center}

where \begin{itemize}
    \item $e = \psi(x) \implies \dot{x} = 0$
    \item $\lambda = \omega(x) \implies \dot{\lambda} = 0$
    \item $\lambda = \varphi(e) \implies \dot{e} = 0$
\end{itemize}

hence, if the non-linear equation \begin{equation}
    \omega(x) = \varphi(\psi(x))
\end{equation} has a root, we have found an equilibrium.

\end{document}