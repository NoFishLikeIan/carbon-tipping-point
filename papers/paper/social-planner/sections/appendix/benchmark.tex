\documentclass[../../main.tex]{subfiles}
\begin{document}

\section{Stochastic Tipping Benchmark Model} \label{appendix:benchmark}

This appendix introduces a benchmark model with stochastic tipping. The stochastic tipping model is a widely used in the economic literature to approximate tipping points in the climate dynamics (e.g. \citealt{hambel_optimal_2021}). Comparing the model developed in this paper with the stochastic tipping model allows us to determine if and how the optimal abatement differ and, as a consequence, what the approximation misses.

To establish a meaningful benchmark, I will assume that the contribution of temperature to forcing (\ref{eq:forcing}) is given by \begin{equation}
    r^{s}_T(T) \coloneqq S_0 (1 - \lambda_1) - \eta\sigma T^4.
\end{equation} This model has no tipping point as $\lambda(T) \equiv \lambda_1$. Stochastic tipping, as commonly modelled in the literature, is introduced  as a jump process $J$ with arrival rate $\pi(T)$ and intensity $\Theta(T)$, both increasing in temperature. Intuitively, as temperature rises, the risk of tipping $\pi(T)$ and the size of the temperature increase $\Theta(T)$ grow. Then temperature dynamics in the Stochastic Tipping model follow \begin{equation}
    \epsilon \dd{T} = \big(r^{s}(T)  + g(m) \big) \dd{t} + \sigma_x \dd{w}^{s} + \Theta(T_t) \dd{N}.
\end{equation} Following \cite{hambel_optimal_2021}, the calibrated arrival rate and temperature increase are calibrated as \begin{align}
    \pi(T) &=  -\frac{1}{4} + \frac{0.95}{1 + 2.8 e^{-0.3325 (T - T^{\mathrm{P}})}} \text{ and } \\
    \Theta(T) &= -0.0577 + 0.0568 (T - T^{\mathrm{P}})-0.0029(T - T^{\mathrm{P}})^2.
\end{align}

\end{document}