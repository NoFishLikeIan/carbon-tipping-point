\documentclass[../../main.tex]{subfiles}
\begin{document}
\section{Solution to Maximisation} \label{appendix:solution}

\subsection{Simplifying Assumptions} \label{appendix:assumptions}

For computational purposes, it is convenient to make some simplifying assumption to reduce the dimensionality of the state space.

\subsubsection{Decay Rate of Carbon}
 
The calibrated carbon decay $\delta_m$, as a function of the carbon stored in sinks $N$, is illustrated in Figure (\ref{fig:decay}). The calibration assumes a functional form \begin{equation}
    \delta_m(N) = a_{\delta} e^{-\left(\frac{N - c_{\delta}}{b_{\delta}}\right)^2},
\end{equation} for parameters $a_{\delta}, b_{\delta}, c_{\delta}$.

\begin{figure}[H]
    \centering
    \input{\plotpath/decay.tikz}
    \caption{Estimated decay of carbon $\delta_m$ as a  function of the carbon stored in sinks $N$.}
    \label{fig:decay}
\end{figure}

To simplify matters I will assume that the amount of carbon sinks present in the atmosphere is a constant fraction of the concentration in the atmosphere, $N = \frac{N_0}{M_0} M$. Using this setup, under a business-as-usual emission scenario, the decay of carbon follows the path in Figure \ref{fig:decaypath}.

\begin{figure}[H]
    \centering
    \input{\plotpath/decaypathfig.tikz}
    \caption{Estimated decay of carbon $\delta_m$ under the business as usual emission scenario $M^{b}$. Each marker is the decay after every decade.}
    \label{fig:decaypath}
\end{figure}

\subsection{Approximating Markov Chain} \label{appendix:approximating-markov-chain}

Th numerical method employed here adapts that presented in \citep{kushner_numerical_2001}. First, we define a suitably large domain for the state variables $\mathcal{X} \subseteq \mathbb{R}^3$ and let $x = (T, m, y) \in \mathcal{X}$ be the state vector. Then, let $u = (\chi, \alpha) \in \mathcal{U} \coloneqq [0, 1] \times [0, \gamma^b]$ be the vector of controls. Then we can define the operator \begin{equation}
    \begin{split}
        \mathcal{L}_t^u = \; &\frac{\mu(T, m)}{\epsilon} \frac{\partial}{\partial T} + \big(\varrho + \phi(\chi) - \delta_k - d(T) - A \beta(\alpha) \big) \frac{\partial}{\partial y} + \\ 
        &(\gamma^b - \alpha)\frac{\partial}{\partial m} + \frac{(\sigma_T / \epsilon)^2}{2} \frac{\partial^2}{\partial T^2} + \frac{\sigma^2_k}{2} \frac{\partial^2}{\partial y^2} 
    \end{split}
\end{equation} such that the value functional at time $t$ satisfies \begin{equation} \label{eq:operator-definition}
    -\partial_t V_t =  \sup_{u} \; \mathcal{L}_t^u V_t + f(\chi, y, V_t).
\end{equation}

We seek to define a Markov chain consistent with (\ref{eq:operator-definition}), over a finite grid in the unit cube \begin{equation}\Omega_h = \{0, h, 2h \ldots 1 - h, 1\}^3.\end{equation} First we define the state dynamics over the unit cube by letting $\tilde{X} = X / \lvert\mathcal{X}\rvert$ where \begin{equation}
    \mathcal{X} = [T^p, T^p + \Delta T] \times [m_0, \overline{m}] \times [y_0, \overline{y}]
\end{equation} and defining the dynamics \begin{equation}
    d \tilde{X} = \omega(t, X, u) \; \dd{t} + \Sigma \; \dd{w}
\end{equation} where \begin{equation}
    \omega(t, X, u) = \begin{pmatrix}
        \mu(T, m) / \epsilon \Delta T \\
        (\gamma^{b} - \alpha) / (\overline{m} - m_0) \\
        \big(\varrho + \phi(\chi) - \delta_k - d(T) - A \beta(\alpha) \big) / (\overline{y} - y_0)
    \end{pmatrix} 
\end{equation} and \begin{equation}
    \Sigma = \begin{pmatrix}
        \sigma_T / \epsilon \Delta T  & 0 & 0 \\
        0 & 0 & 0 \\
        0 & 0 & \sigma_K / (\overline{y} - y_0)\\
    \end{pmatrix}.
\end{equation}

For a given state $X_i$ we can now define the transition probabilities. Let \begin{equation}
    Q(X_i) = \left(\frac{\sigma_T}{\epsilon \Delta T}\right)^2 + \left(\frac{\sigma_K}{\overline{y} - y_0}\right)^2 + h \max_u \; \lvert \omega(t, X_i, u)  \rvert
\end{equation} then \begin{equation}
    p(X_i, X_i \pm h \Delta T) = \frac{\frac{\sigma^2_T}{2(\epsilon \Delta T)^2} + h \; \omega^{\pm}_T(t, X_i, u)}{Q(X_i)}.
\end{equation}

Finally, let \begin{equation}
    V^h(t, X) = \sum_{\tilde{X}} p(X, \tilde{X}) V^h(\tilde{X}) + \frac{1}{1 - \theta} \left(e^{-\rho \Delta t} ((1 - \theta) V^h(t, X))^{\frac{1 - 1 / \psi}{1 - \theta}} + \Delta t C^{1 - \frac{1}{\psi}} \right)^{\frac{1 - \theta}{1 - 1 / \psi}}.
\end{equation} \citep{kushner_numerical_2001} have shown that the transitional probabilities $p$ form a consistent Markov chain and that $V^h \to V$ as $h \to 0$.

\subsection{Post-transition phase}

We assume that at some point in the future $\tau \gg 0$, the abatement rate is equal to emission growth, $\gamma^b = \alpha$, and technological progress caps, $\varrho = 0$, such that the state variables evolve according to dynamics \begin{align}
    \dd{m} &= 0, \\
    \epsilon \; \dd{T} &= \mu(T, m) \; \dd{t} + \sigma_T \; \dd{w_1} \text{ and } \\
    \d y &= \big( \phi(\chi) - \delta_k^p - d(T) \big) \; \dd{t} + \sigma_k \; d w_2.
\end{align} I call this the \textit{post-transition} phase. We can then compute a steady state value function $V_t \eqqcolon \overline{V}$ for all $t \geq \tau$, which satisfies the Hamilton-Bellman-Jacobi equation \begin{equation}
    0 = \overline{\mathcal{L}}^\chi \; \overline{V} + f(\chi, y, \overline{V}),
\end{equation} where \begin{equation}
    \begin{split}
        \overline{\mathcal{L}}^\chi = \; &\frac{\mu(T, m)}{\epsilon} \frac{\partial}{\partial T} + \Big(\phi(\chi) - \delta_k - d(T) \Big) \frac{\partial}{\partial y} + \frac{(\sigma_T / \epsilon)^2}{2} \frac{\partial^2}{\partial T^2} + \frac{\sigma^2_k}{2} \frac{\partial^2}{\partial y^2}.
    \end{split}
\end{equation}


\subsection{Separable Value Function}

The Markov Chain approximation yields the Bellman recurrence \begin{equation}
    \begin{split}
        (1 - \theta) \; V_{t}(T_t, m_t, Y_t) = \inf_{\chi, \alpha} \; \Bigg(&(1 - \beta_{\Delta t} )\frac{\Delta t}{1 + \rho \Delta t}  (\chi Y_t)^{1 - \frac{1}{\psi}} \\
        + \; &\beta_{\Delta t}  \E_{\chi, \alpha} \left[(1 - \theta)V_{t + \Delta t}(T_{t + \Delta t}, m_{t + \Delta t}, Y_{t + \Delta t})  \right]^{\frac{1 - \frac{1}{\psi}}{1 - \theta}} \Bigg)^{\frac{1 - \theta}{1 - \frac{1}{\psi}}}.
    \end{split}
\end{equation} where $\beta_{\Delta t} \coloneqq e^{-\rho \Delta t}$% FIXME: Fix overfull

Now suppose $(1 - \theta) \; V_{t}(T_t, m_t, Y_t) \equiv Y_t^{1 - \theta} F_t(T_t, m_t)$ for some function $F$. Then we can write \begin{equation}
    \begin{split}
        Y_t^{1 - \theta} F_t(T_t, m_t) &= \inf_{\chi, \alpha} \; \left((1 - \beta_{\Delta t}) \; \frac{\Delta t}{1 + \rho \Delta t}  (\chi Y_t)^{1 - \frac{1}{\psi}}
        + \; \beta_{\Delta t}  \E_{\chi, \alpha} \left[Y_{t + \Delta t}^{1 - \theta} F_{t + \Delta t}(T_{t + \Delta t}, m_{t + \Delta t})\right]^{\frac{1 - \frac{1}{\psi}}{1 - \theta}} \right)^{\frac{1 - \theta}{1 - \frac{1}{\psi}}} \\
        F_t(T_t, m_t) &= \inf_{\chi, \alpha} \; \left((1 - \beta_{\Delta t}) \; \frac{\Delta t}{1 + \rho \Delta t}  \chi^{1 - \frac{1}{\psi}}
        + \; \beta_{\Delta t}  \E_{\chi, \alpha} \left[\left(\frac{Y_{t + \Delta t}}{Y_t}\right)^{1 - \theta} F_{t + \Delta t}(T_{t + \Delta t}, m_{t + \Delta t}) \right]^{\frac{1 - \frac{1}{\psi}}{1 - \theta}}  \right)^{\frac{1 - \theta}{1 - \frac{1}{\psi}}} \\
        &= \inf_{\chi, \alpha} \; \left((1 - \beta_{\Delta t}) \; \frac{\Delta t}{1 + \rho \Delta t}  \chi^{1 - \frac{1}{\psi}}
        + \; \beta_{\Delta t}  \left( \delta^y_{\chi, \alpha} \E_{\alpha}F_{t + \Delta t}(T_{t + \Delta t}, m_{t + \Delta t}) \right)^{\frac{1 - \frac{1}{\psi}}{1 - \theta}}\right)^{\frac{1 - \theta}{1 - \frac{1}{\psi}}} 
    \end{split}
\end{equation} where \begin{equation}
    \delta^y_{\chi, \alpha} \coloneqq \E_{\chi, \alpha} \left[ \left(\frac{Y_{t + \Delta t}}{Y_t}\right)^{1 - \theta} \right].
\end{equation} Write \begin{equation}
    \frac{Y_{t + \Delta t}}{Y_t} = 1 + \mu_y(T_t, \chi_t, \alpha_t) \Delta t + \sigma_y \sqrt{\Delta t} z
\end{equation} where $z\sim \mathcal{N}(0, 1)$. Then \begin{equation}
    \begin{split}
        \left(\frac{Y_{t + \Delta t}}{Y_t}\right)^{1 - \theta} = \; &1 + \left(1 - \theta\right) \mu_y(T_t, \chi_t, \alpha_t) \Delta t + \left(1 - \theta\right) \sigma_y \sqrt{\Delta t} z \\
        - &\frac{\theta}{2} \left(1 - \theta\right) \sigma_y^2 z^2 \Delta t + o(\Delta t^{3 / 2}),
    \end{split}
\end{equation} hence \begin{equation}
    \delta^y_{\chi, \alpha} = 1 + \Delta t \left(1 - \theta \right) \left(\mu_y(T_t, \chi_t, \alpha_t) - \frac{\theta}{2} \sigma_y^2 \right) + \E[o(\Delta t^{3 / 2})]
\end{equation}

\end{document}