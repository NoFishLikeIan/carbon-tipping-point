\documentclass[../../main.tex]{subfiles}
\begin{document}
\section{Solution to Maximisation} \label{appendix:solution}

This appendix deals with the solution of the maximisation problem (\ref{eq:social-objective}).

\subsection{Simplifying Assumptions on the Decay Rate of Carbon} \label{appendix:assumptions}
 
To reduce the state space, following \cite{hambel_optimal_2021}, I make an assumption on the decay rate of carbon. The calibrated carbon decay $\delta_m$, as a function of the carbon stored in sinks $N_t$, is illustrated in Figure \ref{fig:decay}. The calibration assumes a functional form \begin{equation}
    \delta_m(N_t) = a_{\delta} e^{-\left(\frac{N_t - c_{\delta}}{b_{\delta}}\right)^2},
\end{equation} for parameters $a_{\delta}, b_{\delta}, c_{\delta}$.

\begin{figure}[H]
    \centering
    \input{\plotpath/decay.tikz}
    \caption{Estimated decay of carbon $\delta_m$ as a  function of the carbon stored in sinks $N_t$.}
    \label{fig:decay}
\end{figure}

I assume that the amount of carbon sinks present in the atmosphere is a constant fraction of the concentration in the atmosphere, $N_t = \frac{N_0}{M_0} M_t$. Abusing notation, I henceforth write $\delta_m(M_t)$ for the decay rate. Using this setup, under a business-as-usual emission scenario, the decay of carbon follows the path in Figure \ref{fig:decaypath}.

\begin{figure}[H]
    \centering
    \input{\plotpath/decaypathfig.tikz}
    \caption{Estimated decay of carbon $\delta_m$ under the business as usual emission scenario $M^{b}$. Each marker is the decay after every decade.}
    \label{fig:decaypath}
\end{figure}

\subsection{Hamilton-Jacobi-Bellmann equation} 

Using the assumption from Appendix \ref{appendix:assumptions}, the value function $V$ at time $t$ depends only on temperature $T_t$, log-carbon concentration $m_t$ and output $Y_t$. This satisfies the Hamilton-Jacobi-Bellmann equation \begin{equation} \label{eq:hjb}
    \begin{split}
        -\partial_t V = \sup_{\chi, \alpha} \; &f(\chi Y, V) + \partial_m V \; (\gamma^b_t - \alpha) + \partial_m^2 V \; \frac{\sigma_m^2}{2}  \\ 
        + \; &\partial_Y V (\varrho + \phi(\chi) - d(T) - \beta_t(\varepsilon(\alpha))) +  \partial_k^2 V \; \frac{\sigma_k^2}{2} \\
        + \; &\partial_T V \; \frac{r(T) + g(m)}{\epsilon} + \partial_T^2 V \; \frac{(\sigma_T / \epsilon)^2}{2}. 
    \end{split}
\end{equation}

It is easy to check that the ansatz \begin{equation} \label{eq:ansatz}
    V_t(T, m, Y) = \frac{Y^{1 - \theta}}{1 - \theta} F_t(T, m) 
\end{equation} satisfies (\ref{eq:hjb}). 

\subsection{Approximating Markov Chain} \label{appendix:approximating-markov-chain}

The Hamilton-Jacobi-Bellmann equation (\ref{eq:hjb}) is solved for $F$ by adapting the method proposed in \cite{kushner_numerical_2001}. The idea is to discretise the state space of $T$ and $m$ and compute time dependent intervals $\Delta t(T, m)$. Then, constructing a Markov chain $\mathcal{M}$ over the discretised space, parametrised by some small step size $h$. Then we compute a discretised value function $F^h$ with the property that $F^h \to F$ as $h \to 0$.

Given an $h$, construct a grid \begin{equation}
    \Omega_h = \{0, h, 2h, \ldots, 1 - h, 1\}^2,
\end{equation} over the unit cube. This grid covers a suitable subset of the state space \begin{equation}
    \mathcal{X} \coloneqq [T^p, T^p + \Delta T] \times [m^p, m^p + \Delta m] 
\end{equation} where $\Delta m$ is chosen such that $T^p + \Delta T$ is stable at $m^p + \Delta m$.

Using the ansatz (\ref{eq:ansatz}) we can define a discrete value function $F^h_t$ over the grid such that $F^h_t \to F_t$ as $h \to 0$ over $\chi$, that is \begin{equation} \label{eq:recursive-markov-chain}
    \begin{split}
        F_t^h(T_t, m_t) = \min_{\chi, \alpha} \; \Bigg( &\left(1 - e^{-\rho \Delta t}\right) \chi^{1 - \frac{1}{\psi}} + \\
        &e^{-\rho \Delta t} \left(\delta_y(\chi) \; \mathbb{E}_{t, \mathcal{M}(\alpha)}  F_{t + \Delta t}^h(T_{t + \Delta t}, m_{t + \Delta t})\right)^{\frac{1 - \frac{1}{\psi}}{1 - \theta}} \Bigg)^{\frac{1 - \theta}{1 - \frac{1}{\psi}}}
    \end{split}
\end{equation} where \begin{equation} 
    \begin{split}
        \delta_y(\chi) &\coloneqq \mathbb{E}_{t} \left[\left(\frac{Y_{t + \Delta t}}{Y_t}\right)^{1 - \theta}\right] \\
        &= 1 + \Delta t (1 - \theta) \left(\varrho + \phi(\chi) - d(T_t) - \frac{\theta}{2} \sigma^2_k \right) + \mathbb{E}_{t} \big[ o(\Delta t^{\frac{3}{2}}) \big].
    \end{split}
\end{equation} and $\mathbb{E}_{t, \mathcal{M}(\alpha)}$ is the expectation with respect to the Markov chain $\mathcal{M}(\alpha)$ over the grid. This can be constructed, given a step size $h$, as follows. Introduce the normalising factor \begin{equation}
    Q_t(T, m, \alpha) \coloneqq \left(\frac{\sigma_T}{ \epsilon \Delta T}\right)^2 + \left(\frac{\sigma_m}{\Delta m}\right)^2 + h \left\lvert \frac{r(T) + g(m)}{\epsilon \Delta T} \right\rvert + h \left\lvert \frac{\gamma^b_t - \alpha}{\Delta m} \right\rvert.
\end{equation} Then the probabilities of moving from a point $(T, m)$ of the grid to an adjacent point are given by \begin{align}
    p(T \pm h \Delta T, m \mid T, m) &\propto \frac{1}{2} \left(\frac{\sigma_T}{ \epsilon \Delta T}\right)^2 + h \left(\frac{r(T) + g(m)}{\epsilon \Delta T}\right)^{\pm} \text{ and } \\
    p(T, m \pm h \Delta m \mid T, m) &\propto \frac{1}{2} \left(\frac{\sigma_m}{\Delta m}\right)^2 + h \left(\frac{\gamma^b_t - \alpha}{\Delta m}\right)^{\pm}
\end{align} where $(\cdot)^{+} \coloneqq \max\{\cdot, 0\}$ and $(\cdot)^{-} \coloneqq -\min\{\cdot, 0\}$. One can readily check that this is a well defined probability measure. Finally, the time step is given by \begin{equation}
    \Delta t = h^2 / Q_t(T, m, \alpha),
\end{equation} which satisfies $\Delta t \to 0$ as $h \to 0$.

Then, as the aggregator used in (\ref{eq:recursive-markov-chain}) converges to $f$ (\ref{eq:aggregator}) \citep{epstein_substitution_1989}, the chain described here satisfies the convergence properties outlined in \cite{kushner_numerical_2001}, we have $F^h_t \to F_t$ as $h \to 0$.

The Markov chain defined above allows to derive $F^h_{t}(T_{t}, m_{t})$ from the subsequent $F^h_{t + \Delta t}(T_{t + \Delta t}, m_{t + \Delta t})$. This requires a terminal condition $\bar{F}^h(T_\tau, m_\tau) \coloneqq F^h_{\tau}(T_{\tau}, m_{\tau})$. To derive this, assume that at some point in a far future $\tau \gg 0$, the abatement is free and all emissions are abated, $\gamma^b = \alpha$, such that $\dd{m} = \sigma_m \dd{W}_m$. Then we construct an equivalent, control independent, Markov chain $\bar{\mathcal{M}}$ as above for \begin{equation}
    \begin{split}
        \bar{F}^h(T_t, m_t) = \min_{\chi} \; \Bigg( &\left(1 - e^{-\rho \Delta t}\right) \chi^{1 - \frac{1}{\psi}} + \\
        &e^{-\rho \Delta t} \left(\delta_y(\chi) \; \mathbb{E}_{t, \bar{\mathcal{M}}} \bar{F}^h(T_{t + \Delta t}, m_{t + \Delta t})\right)^{\frac{1 - \frac{1}{\psi}}{1 - \theta}} \Bigg)^{\frac{1 - \theta}{1 - \frac{1}{\psi}}}.
    \end{split}.
\end{equation} This is now a fixed point equation for $\bar{F}$ which can be solved by value or policy function iteration.

\end{document}