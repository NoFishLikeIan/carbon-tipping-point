\documentclass[../../main.tex]{subfiles}
\begin{document}

\section{Calibration and Parameters} \label{appendix:calibration}

This section summarises the parameters for the preferences, economy, and climate model and discusses the calibration strategy.

The following Table \ref{table:preferences} illustrates the preferences parameters used throughout the paper. There is no consensus in the literature on preference parameters. In line with previous literature focusing on recursive preferences, I set relative risk aversion $\theta = 10$ \citep{ackerman_epsteinzin_2013,crost_optimal_2013,lontzek_stochastic_2015} and the time preference parameter $\rho = 1.5\%$ \citep{nordhaus_estimates_2014}. There is no consensus on whether the elasticity of intertemporal substitution $\psi$ ought to be larger or smaller than unity, with the aforementioned papers using values $\psi \in [0.75, 1.5]$. In this paper, I use $\psi = 1.5$ for the benchmark model and test the robustness of the results to $\psi = 0.75$.

\begin{table}[htbp]
    \centering
    \begin{tabular}{ |p{1cm}||p{3cm}|p{8cm}|}
        \hline
        \multicolumn{3}{|c|}{Preferences} \\
        \hline
        $\rho$ & $1.5\%$ & Time preference \\
        $\theta$ & $10$ & Relative risk aversion \\
        $\psi$ & $1.5$ & Elasticity of intertemporal substitution \\
        \hline
    \end{tabular}
    \caption{}
    \label{table:preferences}
\end{table}

Table \ref{table:economy} summarises the parameters of the economy model. 

To abate a fraction $\epsilon$ of emissions vis-a-vis the BAU scenario requires a fraction \begin{equation}
    \beta(\varepsilon) = \varepsilon^2 
    \omega_0 e^{-\omega_r t}
\end{equation} of GDP. Following \citep{nordhaus_revisiting_2017}, we assume that abating all emissions ($\epsilon = 1$) in 2020 costs 11\% of GDP while doing so in 2100 costs 2.7\% of GDP. This yields the calibrated $\omega_0$ and $\omega_r$. % TODO: Write this out

\begin{table}[htbp] 
    \centering
    \begin{tabular}{ |p{1cm}||p{3cm}|p{8cm}|}
        \hline
        \multicolumn{3}{|c|}{Economy} \\
        \hline
        $\omega_0$ & $11 \%$ & GDP loss required to fully abate today\\
        $\omega_r$ & $2.7 \%$ & Rate of abatement cost reduction \\
        $\varrho$ & $0.9 \%$ & Growth of TFP \\
        $\kappa$ & $6.32 \%$ & Adjustment costs of abatement technology \\
        $\delta_k$ & $0.0116$ & Initial depreciation rate of capital \\
        $\xi$ & $0.00026$ & Coefficient of damage function \\
        $\nu$ & $3.25$ & Exponent of damage function \\
        $A_0$ & $0.113$ & Initial TFP \\
        $Y_0$ & $75.8$ & Initial GDP \\
        $\sigma_k$ & $0.0162$ & Variance of GDP \\
        $\tau$ & $500$ & Steady state horizon  \\
        \hline
    \end{tabular}
    \caption{}
    \label{table:economy}
\end{table}

Table \ref{table:climate} summarises the parameters of the climate model.

\begin{table}[htbp]
    \centering
    \begin{tabular}{ |p{1cm}||p{4cm}|p{7cm}|}
        \hline
        \multicolumn{3}{|c|}{Climate} \\
        \hline
        $T_0$ & $288.56$ [K] & Initial temperature \\
        $T^{\mathrm{p}}$ & $287.15$ [K] & Pre-industrial temperature \\
        $M_0$ & $410$ [p.p.m.] & Initial carbon concentration \\
        $M^{\mathrm{p}}$ & $280$ [p.p.m.] & Pre-industrial carbon concentration \\
        $N_0$ & $286.65543$ [p.p.m.] & Initial carbon in sinks \\
        $\sigma_T$ & $1.5844$ & Volatility of temperature \\
        $S_0$ & $342$ [W / m²] & Mean solar radiation \\
        $\epsilon$ & $15.844$ [J / m² K year] & Heat capacity of the ocean \\
        $\eta$ & $5.67e-8$ & Stefan-Boltzmann constant  \\
        $G_1$ & $20.5$ [W / m²] & Effect of CO$_2$ on radiation budget \\
        $G_0$ & $150$ [W / m²] & Pre-industrial GHG radiation budget \\
        \hline
    \end{tabular}
    \caption{}
    \label{table:climate}
\end{table}

\begin{table}[htbp]
    \centering
    \begin{tabular}{|p{1cm}||p{4cm}|p{7cm}|}
        \hline
        \multicolumn{3}{|c|}{Non-linearity} \\
        \hline 
        $\Delta T$ & $1.8$ [K] & temperature inflection point \\
        $\lambda_1$ & $31\%$ & Initial radiation reflected \\
        $\Delta\lambda$ & . & . \\
        \hline
    \end{tabular}
\end{table}

\end{document}