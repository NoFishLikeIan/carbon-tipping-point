\documentclass[../../main.tex]{subfiles}

\begin{document}
\section{Distribution of temperature} \label{appendix:density}

This section derives the steady state density of temperature given a fixed carbon concentration. The density is then use to calibrate the climate sensitivity and the initial conditions.

The Fokker-Planck equation for the density of temperature $p_t$ is \begin{equation}
    \partial_T \left\{ \frac{\mu(T, m)}{\epsilon} \; p_t(T) + \frac{1}{2} \left(\frac{\sigma_T}{\epsilon}\right)^2 p_t^\prime(T) \right\} = 0.
\end{equation} The steady state temperature $\overline{p}$ then satisfies the differential equation \begin{equation}
    \frac{\mu(T, m)}{\epsilon} \; \overline{p}(T) + \frac{1}{2} \left(\frac{\sigma_T}{\epsilon}\right)^2 \overline{p}^\prime(T) = 0.
\end{equation} One can readily checked that this is solved by  \begin{equation}
    \overline{p}(T) \propto \exp\left( -\frac{V(T, m)}{\sigma^2_T / 2\epsilon^2} \right), 
\end{equation} where \begin{equation}
    \begin{split}
        V(T, m) &\coloneqq  g(m) T + \int r(T) \dd{T} \\
        &= g(m) T + S_0 (1 - \lambda_1) T - \frac{\eta}{5} T^5 + S_0 \Delta\lambda \log(1 + \exp(T - T_i)).
    \end{split}
\end{equation}

\end{document}