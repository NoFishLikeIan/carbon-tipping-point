\documentclass[../../main.tex]{subfiles}

\begin{document}
\section{Steady State Density derivation} \label{appendix:density}

The Fokker-Planck equation for the density of temperature $p$ is \begin{equation}
    \partial_T \left\{ \frac{1}{\epsilon} \mu(T, m) p(T) + \frac{\sigma^2_T}{2\varepsilon^2} p^\prime(T) \right\} = 0,
\end{equation} such that, the steady state temperature $\overline{p}$ satisfies the ODE \begin{equation}
    \frac{1}{\epsilon} \mu(T, m) \overline{p}(T) + \frac{\sigma^2_T}{2\varepsilon^2} \overline{p}^\prime(T) = 0,
\end{equation} which has solutions \begin{equation}
    \overline{p}(T) \propto exp\left( -\frac{V(T, m)}{\sigma^2_T / 2\epsilon^2} \right), 
\end{equation} where \begin{equation}
    \begin{split}
        V(T, m) = &(\mu_m(m) + (1 - \lambda_1)S_0) T - \frac{\eta}{5} T^5 + \\
        &S_0 (\lambda_1 - \lambda_2) \log(1 + \exp(T - T_i)).
    \end{split}
\end{equation}


\subsection{Abatement Costs}

To abate a fraction $e$ of emissions vis-a-vis the BAU scenario requires a fraction \begin{equation}
    \beta(\varepsilon) = \varepsilon^2 
    ; \omega_0 e^{-\omega_r t}
\end{equation} of GDP. Following \citep{nordhaus_revisiting_2017}, we assume that abating all emissions ($e = 1$) in 2020 costs 11\% of GDP while doing so in 2100 costs 2.7\% of GDP. This yields the calibrated $\omega_0$ and $\omega_r$

\end{document}