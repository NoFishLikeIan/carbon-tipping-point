\documentclass[../../main.tex]{subfiles}
\begin{document}

As the average world temperature rises, due to the emissions of greenhouse gases from human economic activities, positive feedbacks can push the climate system through critical thresholds, known as tipping points, into a regime of higher temperature. Reverting back the climate system to the current regime can be hard, if not impossible. This risk affects the trade-off between the economic gains from emissions and the damages such emissions impose on the economy. Unfortunately, there is a fundamental uncertainty around these tipping points. In this paper, I study the societal costs of late or insufficient abatement, driven by the misestimation of the tipping point and the climate system. To do so, I compute the socially optimal abatement policy in an integrated assessment model under three climate specification. A climate model with an imminent tipping point, one with a remote tipping point, and, finally, a climate model with stochastic tipping, as commonly used in the literature (see e.g. \citealt{van_der_ploeg_climate_2018,lemoine_ambiguous_2016,hambel_optimal_2021}). I then compute the social costs associated with being wrong: abating under the assumption of a remote tipping point or a stochastic tipping point, despite the tipping point being imminent.

% Why you are wrong, matters.

\iffalse

The positive feedback affects the dynamics of temperature, and as a consequence optimal abatement, in three ways. First, it introduces a tipping point in the temperature rise. That is, there is a level of carbon concentration after which temperature start increasing rapidly to a new, hotter steady state. Second, while approaching the tipping point, positive temperature shocks become more persistent than negative ones. Third, the abatement levels necessary to revert back temperature to its pre-tipping levels become discontinuously large. 

The importance of modelling precise climate dynamics and tipping points when determining optimal emission paths has been increasingly recognised in economics \citep{van_den_bremer_risk-adjusted_2021,dietz_economic_2021,dietz_are_2020,taconet_social_2021,lontzek_stochastic_2015}. Previous approaches have mostly focused on the stochastic nature of tipping points, by modelling temperature dynamics \citep{dietz_economic_2021} or damages \citep{lontzek_stochastic_2015} as jump processes, with arrival rates increasing in emissions. Yet, many tipping points in the climate system are caused by bifurcations \citep{ashwin_extreme_2020,ashwin_tipping_2012}. In this paper, I show that introducing this class of tipping points in an integrated assessment model yields similar predictions in terms of aggregate emissions, but prescribes much steeper reduction of emissions to keep the risk of tipping low.

To tease out this difference, I study an AK-model in which increases in temperature, beyond pre-industrial levels, reduce economic growth (as in \cite{pindyck_economic_2013} and \cite{hambel_optimal_2021}). This modelling choice, as opposed to having temperatures wipe-out a fraction of the capital stock, as in \citeauthor{nordhaus_estimates_2014} (\citeyear{nordhaus_estimates_2014,nordhaus_revisiting_2017}), is motivated by recent evidence on the role of temperature in reducing economic growth and productivity \citep{burke_global_2015, dietz_growth_2019}.

\fi

\end{document}