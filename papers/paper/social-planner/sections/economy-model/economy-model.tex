\documentclass[../../main.tex]{subfiles}
\begin{document}

\subsubsection{Capital and Climate Damages}

Output $Y_t$ is the product of the capital stock $K_t$ and its productivity $A_t$ \begin{equation}
Y_t \coloneqq A_t K_t.
\end{equation} Productivity is assumed to grow at a constant rate $\varrho$. Output $Y_t$ can be used for investment in capital $I_t$, abatement expenditures $B_t$, or consumption $C_t$, imposing the constraint \begin{equation} \label{eq:nominal-budget}
    Y_t = I_t + B_t + C_t.
\end{equation} In absence of climate change, $K_t$ depreciates at a rate $\delta_k$ but can be substituted by capital investments $I_t$, which incurs, along with abatement expenditure $B_t$, in quadratic adjustment costs \begin{equation} \label{eq:adjustment-costs}
    \frac{\kappa}{2} \left( \frac{I_t + B_t}{K_t} \right)^2  \; K_t.
\end{equation}

Climate change interacts with the economy by lowering capital growth via damages $d(T_t)$ which are increasing in temperature $T_t$. Following \cite{weitzman_ghg_2012}, I assume the damage function to take the form \begin{equation}
    d(T_t) \coloneqq \xi \; (T_t - T^p)^{\upsilon}
\end{equation} where $T^p$ is the pre-industrial level of temperature (Figure \ref{fig:damage}). \begin{figure}[H]
    \centering
    \input{\plotpath/damagefig.tikz}
    \caption{Damage function with the \cite{weitzman_ghg_2012} calibration.}
    \label{fig:damage}
\end{figure} \noindent This stylised form captures the empirical evidence that under higher temperature levels some forms of capital, particularly in the agricultural \citep{dietz_growth_2019} or manufacturing sectors \citep{dell_temperature_2009}, become more expensive or harder to substitute. A common alternative in the literature is to assume that higher temperatures wipe out part of the capital stocks \citep{nordhaus_optimal_1992}. The comparison between these two assumption is carried out in Appendix (\textbf{?}) and a thorough treatment can be found in \citep{hambel_optimal_2021}. 

Putting the endogenous growth of capital and the climate damages together, the growth rate of capital satisfies \begin{equation} \label{eq:growth-capital:nominal}
    \frac{\dd{K}_t}{K_t} = \left(\frac{I_t}{K_t} - \delta_k - \frac{\kappa}{2} \left( \frac{I_t + B_t}{K_t} \right)^2 \right) \dd{t} - d(T_t) \dd{t} + \sigma_k \dd{W}_k,
\end{equation} where $W_k$ is a Weiner process.

In the following, I link the abatement costs $B_t$ with the abatement rate $\alpha_t$, introduced in the previous section. Introduce $\beta \coloneqq B_t / Y_t$ the fraction of output devoted to abatement. As in \cite{nordhaus_optimal_1992}, I assume this to be quadratic in the fraction of abated emissions $\varepsilon(\alpha_t)$, namely, \begin{equation} \label{eq:abatement-costs}
    \beta_t(\varepsilon(\alpha_t)) = \frac{\omega_t}{2} \; \varepsilon(\alpha_t)^2.
\end{equation} Under this assumption, no abatement is free, as $\beta_t(0) = 0$. At a fixed time $t$, higher abatement rates $\alpha_t$ and hence higher emission reduction vis-a-vis the business as usual scenario $\varepsilon(\alpha_t)$, become increasingly costly at a rate $\omega_t \varepsilon(\alpha_t)$. As time progresses, so does abatement technology and a given abatement objective becomes cheaper. This is modelled by letting the exogenous technological parameter $\omega_t$ decrease over time. There is large uncertainty around the path of $\omega_t$. % TODO: Talk about this

Finally, let \begin{equation}
    \chi_t \coloneqq \frac{C_t}{Y_t}
\end{equation} be the rate of output devoted to consumption. Using the budget constraint (\ref{eq:nominal-budget}) and the two controls $\alpha_t$ and $\chi_t$, the growth rate of capital (\ref{eq:growth-capital:nominal}) can be rewritten as \begin{equation}
    \frac{\dd{K_t}}{K_t} = \Big(\phi_t(\chi_t) - A_t \beta_t(\varepsilon(\alpha_t)) - d(T_t) \Big) \dd{T} + \sigma_k \dd{W}_k
\end{equation} where \begin{equation}
    \phi_t(\chi_t) \coloneqq A_t (1 - \chi_t) - \frac{\kappa}{2} A_t^2 (1 - \chi_t)^2 - \delta_k
\end{equation} is an endogenous growth component, $\beta_t(\varepsilon(\alpha_t))$ is the cost rate of abatement $\alpha_t$, and $d(T_t)$ are the damages from climate change. This formulation makes the trade-off between climate abatement and economic growth apparent. Devoting fewer resources to abatement to pursue higher capital, and hence, output growth, yields higher future temperature and can put the economy in a lower growth path altogether.

Finally, as output $Y_t$ is just the product of capital $K_t$ and productivity $A_t$, its growth rate differs from that of capital just by the growth rate of productivity, hence it satisfies \begin{equation}
    \frac{\dd{Y}_t}{Y_t} = \varrho + \frac{\dd{K}_t}{K_t}. 
\end{equation}

\end{document}