\documentclass[../../main.tex]{subfiles}
\begin{document}

This section introduces the objective of the social planner and the resulting maximisation problem. Societal utility at time $t$ given a state $X_t \coloneqq (T_t, M_t, N_t, Y_t)$ is recursively defined as \begin{equation} \label{eq:social-objective}
    V_t(X_t) = \sup_{\chi, \alpha} \mathbb{E}_t 
    \int_{t}^{\infty} f(\chi_s(X_s) Y_s, V_s(X_s)) \dd{s}
\end{equation} where $\chi$ and $\alpha$ are continuous functions over time and the state space, and $f$ is an Epstein-Zin aggregator \begin{equation} \label{eq:aggregator}
    f(c, u) \coloneqq \frac{\rho}{1 - 1 / \psi} (1 - \theta) u  \left( \left(\frac{c}{((1 - \theta) u)^{\frac{1}{1 - \theta}}}\right)^{1 - 1 / \psi} - 1 \right).
\end{equation} Consumption is integrated into a utility index by means of the Epstein-Zin integrator \citep{duffie_asset_1992}. This aggregator plays a dual role. First, it allows to disentangle the role of relative risk aversion $\theta$, elasticity of intertemporal substitution $\psi$ and the discount rate $\rho$ in determining optimal abatement paths. Second, it circumvents the known paradoxical result that abatement policies becomes less ambitious as society becomes more risk averse \citep{pindyck_economic_2013}. Details on solving the problem (\ref{eq:social-objective}) are given in Appendix \ref{appendix:solution}. 

The paths of consumption and climate abatement resulting from the maximisation problem are then the optimal consumption and abatement paths. 


\end{document}