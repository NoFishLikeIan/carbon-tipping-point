\documentclass[../../main.tex]{subfiles}
\begin{document}

\subsubsection[Carbon Dioxide concentration]{CO$_2$ concentration and carbon sinks}

Emissions from human economic activity $E_t$ increase the average atmospheric concentration of CO$_2$ $M_t$. This, in turn, decays into natural sinks $N_t$. To model the saturation of natural sinks, the decay rate $\delta_m(N_t)$ falls in the quantity of carbon dioxide already stored in the natural sinks $N_t$. Hence, this evolves as \begin{equation}
    \xi_m \dd{N_t} = \delta_m(N_t) M_t \; \dd{t}
\end{equation} where $\xi_m$ is a factor converting quantities from parts-per-million in volume to \unit{Gt} of CO$_2$.

As we are concerned with abatement efforts, vis-à-vis a ``business as usual'' scenario, I rewrite variables in deviation from such scenario. Variables under business as usual are then calibrated using the IPCC SPSS5 projections (\citeyear{intergovernmental_panel_on_climate_change_climate_2023}). Denote by $E^b_t$ the emissions under such scenario, and by $M^b_t$ and $N^b_t$ the resulting carbon in the atmosphere and in natural sinks, respectively. The atmospheric concentration $M^b_t$ under business as usual emissions evolves as \begin{equation} \label{eq:dynamics:carbon-concentration:bau}
    \frac{\dd{M^b_t}}{M^b_t} = \gamma^b_t \dd{t} + \sigma_m \dd{W}_{m, t}
\end{equation} where \begin{equation}
    \gamma^b_t \coloneqq \xi_m \frac{E^b_t}{M^b_t} - \delta_m(N^b_t)
\end{equation} and $W_{m, t}$ is a Wiener process. The business as usual is chosen to be IPCC's SSP5 scenario \citep{kriegler_fossil-fueled_2017}. This scenario describes an energy intensive future, in which fossil fuel usage develops rapidly and little to no abatement takes place. Using this scenario, I then calibrate the implied growth rate of carbon concentration $\gamma_t^{b}$. Figure \ref{fig:bau-growth} shows the results of the calibration (Appendix \ref{appendix:calibration}). The upper figure show s the path of the business as usual growth rate $\gamma^b_t$ and the lower figure shows the implied growth of carbon concentration $M_t^b$. \begin{figure}[htbp]
    \centering
    \input{\plotpath/growthmfig.tikz}
    \caption{Growth rate of carbon concentration in the business as usual scenario $\gamma^b_t$ and median path (solid) of business as usual carbon concentration $M^b_t$ (\ref{eq:dynamics:carbon-concentration:bau}) with 5\% and 95\% confidence intervals (dashed).}
    \label{fig:bau-growth}
\end{figure} \noindent The carbon concentration in this scenario is assumed to grow at an increasingly fast rate until 2080, when the growth rate peaks at around 1.4\%. Thereafter, the growth rate starts declining. Recall that $\gamma^b_t$ is the growth rate of carbon concentration $M^b_t$, which is always positive, hence in the business as usual scenario, carbon concentration is always increasing.

Abatement efforts $\alpha_t$ lower the growth rate of carbon concentration $M_t$ vis-à-vis the business as usual scenario $M^b_t$. Introduce the growth rate of carbon concentration $m_t \coloneqq  \log M_t$. Then, its evolution is given by \begin{equation} \label{eq:dynamics:carbon-concentration}
    \dd{m}_t = \big(\gamma^b_t - \alpha_t \big) \dd{t} + \sigma_m \dd{W}_{m, t}.
\end{equation} By assumption, I assume that negative emissions are not attainable, namely \begin{equation}
    \alpha_t \leq \gamma^b_t + \delta_m(N_t).
\end{equation} This assumption is tested in Appendix (\textbf{?}). \iffalse TODO: talk about this assumption further. \fi Implementing no abatement policy $\alpha_t = 0$ corresponds to a business as usual scenario $M_t \equiv M^b_t$, while implementing a full abatement policy $\alpha_t = \gamma^b_t$ stabilises carbon concentration. Any abatement policy $\alpha_t$ can be implicitly linked back to the corresponding level of emissions by introducing an emission reduction rate $\varepsilon(\alpha_t)$, which keeps tracks of what percentage of emission has been abated \begin{equation} \label{eq:emissivity}
    E_t = \big(1 - \varepsilon(\alpha_t)\big) E^b_t.
\end{equation}

\subsubsection{Temperature}

Earth's radiating balance, in its simplest form, prescribes that an equilibrium temperature $\bar{T}$ is determined by equating incoming solar radiation $S$ with outgoing long-wave radiations $\eta \sigma \bar{T}^4$, where $\sigma$ is the Stefan-Boltzmann constant and $\eta$ is an emissivity rate. Due to the presence of greenhouse gasses, certain wavelengths are scattered and, hence, not emitted\footnote{See \cite{ghil_topics_2012} and \cite{greiner_economic_2005} for a more detailed discussion.}. This introduces an additional radiative forcing $G$ which yields the balance equation $S = \eta \sigma \bar{T}^4 - G$. Focusing on the role of increased CO$_2$, as opposed to other greenhouse gases, we can decompose the greenhouse radiative forcing term $G$ into a constant component $G_0$ and a component which depends on the steady state level of CO$_2$ concentration in the atmosphere $M_t$ with respect to the pre-industrial level $M^p$, such that \begin{equation}
    G \equiv G_0 + G_1 \log(M_t / M^p) = G_0 + G_1 \left(m_t - m^p_t\right).
\end{equation} I introduce a feedback in the temperature by assuming that the absorbed incoming solar radiation is increasing in temperature. This choice can be seen as a stylised model of the ice-albedo feedback \citep{mcguffie_climate_2005, ashwin_tipping_2012}, yet, any positive feedback in temperature dynamics would yield similar interpretations. 
The incoming solar radiation $S$ can be then decomposed into $S_0 \left(1 - \lambda(T_t)\right)$ where the function $\lambda(T_t)$ transitions from a higher $\lambda_1$ to a lower level $\lambda_1 - \Delta \lambda$ via a smooth transition function $L(T_t)$. To control at which level of temperature the transition occurs the transition functions take the form \begin{subequations} \label{eq:albedo-specification}
    \begin{align}
        \lambda(T_t) &\coloneqq \lambda_1 - \Delta \lambda \big(1 - L(T_t)\big) \text{ where } \label{eq:lambda} \\
        L(T_t) &\coloneqq \left(1 + \exp \left(-L_1 \left(T_t - T^p - T^c - \frac{\Delta T}{2}\right)\right)\right)^{-1}
    \end{align} 
\end{subequations} where $T^p$ is the pre-industrial level of temperature, $T^c$ is the level of temperature at which the feedback effect begins, $\Delta T$ is the duration of the transition, and $L_1$ is a speed parameter. 

The costs that society incurs if it behaves under the wrong assumption over $T^c$ are the focus of the subsequent sections. Best estimates from climate sciences are that many such transitions occur for average global temperatures between $1.5^\circ$ and $3^\circ$ over pre-industrial levels \citep{seaver_wang_mechanisms_2023}. Yet, there is large uncertainty around these variables. \cite{ben-yami_uncertainties_2024} show that the uncertainty is so large that estimating critical thresholds $T^c$ from historical data is unfeasible. In this paper, I consider two extreme scenarios: one in which the tipping point is \textit{imminent} $T^c = 1.5^\circ$ and one in which it is \textit{remote} $T^c = 2.5^\circ$. Figure \ref{fig:albedo_coefficient} shows the transition function (\ref{eq:albedo-specification}) under these two scenarios. \begin{figure}[htbp]
    \centering
    \input{\plotpath/albedo.tikz}
    \caption{Coefficient $\lambda(T)$ for different threshold temperatures $T^c \in \{1.5, 3.5\}$}
    \label{fig:albedo_coefficient}
\end{figure} Despite being a highly stylised average model for a complex and spatially heterogeneous process, $\lambda$ captures the core mechanism behind feedback processes in the temperature dynamics. Putting these processes together we can write the two determinants of temperature dynamics: radiative forcing, which only depends on temperature, \begin{equation} \label{eq:forcing}
    r(T_t) \coloneqq S_0 \left(1 - \lambda(T_t)\right) - \eta\sigma T_t^4
\end{equation} and the greenhouse gasses effects, which only depends on the log deviation of atmospheric carbon concentration with respect to its pre-industrial levels, \begin{equation}
    g(m_t) \coloneqq G_0 + G_1 (m_t - m^p).
\end{equation} Putting these two drivers together we can write temperature deviations as \begin{equation} \label{eq:dynamics:temperature}
    \epsilon \dd{T_t} = r(T_t) \dd{t} + g(m_t) \dd{t} + \sigma_T \; dW_{T, t}, 
\end{equation} where $\epsilon$ is the thermal inertia and $W_{T, t}$ is a Wiener process.

\subsubsection{Tipping Points}

The presence of the feedback effect $\lambda$ introduces tipping points in the temperature dynamics. This is illustrated in Figure \ref{fig:bare-nullclines} for a critical temperature $T^c = 2^\circ$. For a given level of carbon concentration $\bar{M}$, the temperature $\bar{T}$ tends towards a steady state satisfying $\mathbb{E} \dd{T}_t = 0$ or equivalently $r(\bar{T}) = - g(\bar{m})$ where $\bar{m} = \log \bar{M}$ (solid line). \begin{figure}[htbp]
    \centering
    \input{\plotpath/bare-nullcline.tikz}
    \caption{Steady states of temperature $T_t$ and carbon concentration $M_t$ for $T^c = 2^\circ$. The solid and dashed lines represent attracting and repelling steady states, respectively.}
    \label{fig:bare-nullclines}
\end{figure} \noindent For low values of carbon concentration $\bar{M} \lesssim 350$, such steady state is unique. As more carbon dioxide gets added to the atmosphere, two additional steady state levels of temperature $\bar{T}$ arise, one stable (upper solid line) and one separating unstable (dashed line). The presence of the additional steady state is hard to detect as the relationship between temperature $T_t$ and carbon concentration $M_t$ is still log-linear. This makes estimating the presence of a tipping point and its threshold $T^c$ complicated. As carbon concentration increases further $M_t \gtrsim 490$ and temperature crosses the critical threshold $T^c$, the old stable and low temperature regime is not feasible any more and only a high stable temperature regime remains. Any increase of carbon concentration at this tipping point would then lead to a rapid increase in temperature to a this new steady state. Crucially, to revert the system back to the lower temperature regime, it is not sufficient to remove just the carbon that caused the system to tip, but society would have to remove all carbon until the only stable steady state is the low temperature, hence back to $M_t \lesssim 350$ (``walking along'' the upper solid line). In the context of the example of an ice-albedo tipping point, at low level of carbon concentration the Earth's ice coverage is large and the albedo coefficient, that is, the percentage of solar radiation reflected by the lighter coloured surface. As temperature rises, and ice melts, the albedo decreases, which further contributes to temperature increases. This positive feedback might push the world into an ``ice-free'' regime with low albedo. To restore ice coverage temperature must decrease further than the increase of the tipping point.

Figure \ref{fig:nullclines} displays how this mechanism impacts the dynamics of temperature under the business as usual scenario for a critical thresholds $T^c$ of  $2^\circ$ (darker) and $3.5^\circ$ (lighter). The lines with the markers show a simulation of temperature $T_t$ (\ref{eq:dynamics:temperature}) and carbon concentration $M_t$ under business as usual (\ref{eq:dynamics:carbon-concentration:bau}). Markers denotes the temperature and carbon concentration every 10 years starting from 2020. For the first three decades, the temperature dynamics are identical. Then suddenly, if the tipping point is closer $T^c = 2^\circ$, temperature grows rapidly to the new steady state and, in 20 years, the system converges to a high temperature regime, before the log-linear relationship between temperature $T_t$ and carbon concentration $M_t$ is established again. This occurs much later whenever the critical temperature $T^c$, and hence the tipping point, are higher. Given the high uncertainty behind $T^c$ and the resulting different temperature dynamics, after introducing the economy in the next section, the paper focuses on the optimal abatement efforts under different levels of $T^c$ and the costs associated with wrongly estimating it. \begin{figure}[htbp]
    \centering
    \input{\plotpath/nullcline.tikz}
    \caption{Steady states of temperature $T_t$ and carbon concentration $M_t$ for $T^c = 2^\circ$ (dark) and $T^c = 3^\circ$ (light). The solid and dashed lines represent attracting and repelling steady states, respectively. The marked line show two simulation of temperature and carbon concentration under the business as usual scenario. Markers denotes the temperature and carbon concentration every 10 years starting from 2020.}
    \label{fig:nullclines}
\end{figure}

% TODO: Talk about the limitations of stochastic tipping
\iffalse
\subsubsection{Critical Slowdown and Early Warning Signals}

\begin{wrapfigure}{r}{0.5\textwidth}
    \centering
    \input{\plotpath/densfig.tikz}
    \caption{Density of temperature shocks at $M = 600$.}
    \label{fig:density}
\end{wrapfigure}
Another crucial difference between the climate dynamics in the presence of a tipping point, vis-à-vis those modelled with a jump process (e.g. \citealt{dietz_economic_2021,hambel_optimal_2021}) is the presence of critical slowdown: as we approach the tipping point, large temperature deviation shocks persist for longer, as ice struggles to reform. To illustrate this effect, Figure \ref{fig:density} shows the density of temperature as carbon concentration approaches the tipping point (derived in Appendix \ref{appendix:density}). For larger levels of albedo loss, large deviations of temperature are more persistent, hence more time are spent in a high temperature regime. This has two relevant implications for optimal emissions. First, it can act as an early warning signal; if one is uncertain about the size of the albedo loss, long periods of high temperature can be used to infer that the loss is high and the system is approaching the tipping point. Second, such repeated and persistent periods of high temperatures generate large economic damages.
\fi
\end{document}