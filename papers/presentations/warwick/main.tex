\documentclass[final]{beamer}

\usepackage[scale=1.24]{beamerposter} % Use the beamerposter package for laying out the poster

\usetheme{confposter} % Use the confposter theme supplied with this template

\setbeamercolor{block title}{fg=ngreen,bg=white} % Colors of the block titles
\setbeamercolor{block body}{fg=black,bg=white} % Colors of the body of blocks
\setbeamercolor{block alerted title}{fg=white,bg=dblue!70} % Colors of the highlighted block titles
\setbeamercolor{block alerted body}{fg=black,bg=dblue!10} % Colors of the body of highlighted blocks
% Many more colors are available for use in beamerthemeconfposter.sty

%-----------------------------------------------------------
% Define the column widths and overall poster size
% To set effective sepwid, onecolwid and twocolwid values, first choose how many columns you want and how much separation you want between columns
% In this template, the separation width chosen is 0.024 of the paper width and a 4-column layout
% onecolwid should therefore be (1-(# of columns+1)*sepwid)/# of columns e.g. (1-(4+1)*0.024)/4 = 0.22
% Set twocolwid to be (2*onecolwid)+sepwid = 0.464
% Set threecolwid to be (3*onecolwid)+2*sepwid = 0.708

\newlength{\sepwid}
\newlength{\onecolwid}
\newlength{\twocolwid}
\newlength{\threecolwid}
\setlength{\paperwidth}{48in} % A0 width: 46.8in
\setlength{\paperheight}{36in} % A0 height: 33.1in
\setlength{\sepwid}{0.024\paperwidth} % Separation width (white space) between columns
\setlength{\onecolwid}{0.22\paperwidth} % Width of one column
\setlength{\twocolwid}{0.464\paperwidth} % Width of two columns
\setlength{\threecolwid}{0.708\paperwidth} % Width of three columns
\setlength{\topmargin}{-0.5in} % Reduce the top margin size
%-----------------------------------------------------------

\usepackage{graphicx}  % Required for including images
\usepackage{subcaption}

\usepackage{booktabs} % Top and bottom rules for tables


% Math
\usepackage{amsmath, amssymb, mathtools, bbm, bm}
\usepackage[makeroom]{cancel}
\definecolor{cbred}{RGB}{139, 38, 53}
\definecolor{cbblue}{RGB}{46, 53, 50}
\definecolor{cbyellow}{RGB}{255, 193, 7}
\definecolor{cbgreen}{RGB}{0, 77, 64}
\definecolor{cbdarkblue}{RGB}{30, 30, 230}


\let\d\relax
\newcommand{\d}[1]{\mathrm{d}#1}
\newcommand{\control}[1]{\bm{#1}}
\newcommand{\T}{\mathbf{T}}
\newcommand{\M}{\mathbf{M}}
\newcommand{\Y}{\mathbf{Y}}

% Diagrams
\usepackage{tikz} 
\usepackage{tikzit}
\usepackage{tikzscale}
\usepackage{pgfplots}
\usepackage{graphicx}
\usepackage{wrapfig}
\usepackage{relsize}


% Graphs
\usepackage{tikz} 
\usepackage{tikzit}
\usepackage{pgfplots}
\usepackage{ifdraft}
\usetikzlibrary{positioning,fit,calc,decorations.pathreplacing,calligraphy, arrows.meta, pgfplots.groupplots}

\usepgfplotslibrary{external} 
\tikzexternalize

\pgfplotsset{compat=1.18}

\newcommand{\X}{{\color{violet} X}}
\newcommand{\includegraphics[width = 0.5\linewidth]}[1]{%  

    \ifdraft{
    \includegraphics[width=0.5\textwidth]{example-image-a}
    }{
    \IfFileExists{#1}{
        \input{#1}
        }{
        \includegraphics[width=0.5\textwidth]{example-image-a}
    }
    }
}

\usepackage[bibencoding=utf8, style=apa, backend=biber, eprint=false]{biblatex}

\addbibresource{../../scc-tipping-points.bib}


% -- TITLE SECTION 
\title{Optimal Emissions and the Nature of Climate Tipping Points} % Poster title

\author{Andrea Titton} % Author(s)

\institute{CeNDEF, University of Amsterdam $\vcenter{\hbox{\includegraphics[width=3cm]{uvalogo_tag_p.eps}}}$}
% -- 

\begin{document}

\addtobeamertemplate{block end}{}{\vspace*{2ex}} % White space under blocks
\addtobeamertemplate{block alerted end}{}{\vspace*{2ex}} % White space under highlighted (alert) blocks

\setlength{\belowcaptionskip}{2ex} % White space under figures
\setlength\belowdisplayshortskip{2ex} % White space under equations

\begin{frame}[t] % The whole poster is enclosed in one beamer frame

\begin{columns}[t] % The whole poster consists of three major columns, the second of which is split into two columns twice - the [t] option aligns each column's content to the top

% First column
\begin{column}{\sepwid}\end{column} 
\begin{column}{\twocolwid}

\textbf{\large The Issue of Simplified Climate Dynamics in Economics} \vspace{1.5em}

\begin{column}{\onecolwid}
    {\color{Econ} \textbf{The Economics Way}} \vspace{1em}

    A tipping point is a \textbf{stochastic} jump $N$ in the temperature $\T$ with size $\epsilon(\T)$ and intensity $\eta(\T)$, both increasing in $\T$.

    \begin{equation*}
        \begin{split}
            \epsilon \; \dd{{\T} = \; &C(\M) \; \dd{t} \;+ \sigma \; \dd{W} \\
            + \; &S + \epsilon(\T) \; \dd{N}
        \end{split}
    \end{equation*}

    \begin{figure}
        \includegraphics[width = 0.5\linewidth]{\plotpath/illjumpfig.tikz}
    \end{figure} 

\end{column}
\begin{column}{\onecolwid}
    {\color{Climate} \textbf{The Climate Science Way}} \vspace{1em}

    A tipping point is a \textbf{bifurcation} in the temperature dynamics, for example, the albedo coefficient $\lambda(\T)$.

    \begin{equation*}
        \begin{split}
            \epsilon \; \dd{{\T} = \; &C(\M) \; \dd{t} \;+ \sigma \; \dd{W} \\
            + \; &S \big(1 - \lambda(\T)\big) \dd{t}
        \end{split}
    \end{equation*}

    \begin{figure}
        \includegraphics[width = 0.5\linewidth]{\plotpath/poster_albedo.tikz}
    \end{figure} 
\end{column}

\vspace{1em}

Two different ways that lead to two different \textbf{dynamics}.

\begin{figure}
    \includegraphics[width = 0.5\linewidth]{\plotpath/poster_baufig.tikz}
\end{figure}

\vspace{1em}
\begin{alertblock}{Research Question}
    \textbf{Do these different dynamics matter for optimal emissions? Roadmap:}
    \begin{itemize} \setlength\itemsep{0.5em}
        \item \textbf{Embed these dynamics in a standard AK economy (\cite{hambel_optimal_2021})}
        \item \textbf{Solve computationally for optimal emission (\cite{kushner_numerical_2001})}
    \end{itemize}
\end{alertblock}

\end{column} % End of outer 1st columns

\begin{column}{\sepwid}\end{column}
\begin{column}{\twocolwid} % Beginning of outer 2nd columns
    
\textbf{\large Economy and Climate Trade-Off}  \begin{equation*}
    \frac{\d \Y}{\Y} = \overbrace{\varrho - \delta}^{\text{endogenous growth}} + \underbrace{\phi(\chi)}_{\text{investment}} - \overbrace{d(\T)}^{\text{climate damages}} - \underbrace{\omega c(\alpha)}_{\text{abatement}}
\end{equation*}


\textbf{\large Planner Problem} \vspace{1em}

\begin{columns}
    \begin{column}{\onecolwid}
        \textbf{State}
        
        \begin{itemize}
            \item GDP $\Y$
            \item Temperature $\T$
            \item Carbon Concentration $\M$
        \end{itemize}
        \textbf{Control}
        \begin{itemize}
            \item Abatement rate vis-à-vis business as usual $\alpha$
            \item Fraction of GDP consumed $\chi$
        \end{itemize}
    \end{column}
    \begin{column}{\onecolwid}
        \textbf{Objective}
        Maximise net-present value of consumption \begin{equation*}
            V_0 = \sup_{\chi, \alpha} \int^\infty_0 f(\chi_t \Y_t, V_t) \; \dd{t}
        \end{equation*} where $f$ is Epstein-Zin aggregator (\citeyear{epstein_substitution_1989})
    \end{column}
\end{columns}

\begin{columns}
    \begin{column}{\onecolwid}
        \textbf{\large Solution} \vspace{1em} 
        \begin{itemize}
            \item \textbf{Analytical} Linked tipping point and optimal emissions using perturbation theory (\cite{van_den_bremer_risk-adjusted_2021})
            \item \textbf{Computational} Constructed a novel markov chain $\mathcal{M}$ with heterogeneous step size $\Delta(T, m)$ \begin{equation*}
                \begin{split}
                    V_t = \sup \; &(1 - e^{-\rho \Delta(T, m)}) c^{{1 - 1 / \psi}} + \\ 
                    &e^{-\rho \Delta(T, m)} ((1 - \theta) \mathbb{E}_{\mathcal{M}} V_{t + \Delta(T, m)})^{\frac{1 - 1 / \psi}{1 - \theta}}
                \end{split}
            \end{equation*}
        \end{itemize}

    \end{column}
    \begin{column}{\onecolwid}
        \textbf{\large Optimal Emissions} \vspace{1em} 
        \begin{figure}
            \includegraphics[width = \linewidth]{epoc-figure10.pdf}
        \end{figure}
    \end{column}
\end{columns}

\begin{alertblock}{Key Takeaways}
    \begin{itemize} \setlength\itemsep{0.5em}
        \item \textbf{The dynamics matter a lot: net-zero 20 years earlier}
        \item \textbf{Stochastic tipping is a good approximation for steep temperature increases without tipping}
        \item \textbf{Computational and analytical tools to deal with nonlinearities exist: no excuses!}
    \end{itemize}
\end{alertblock}

\begin{columns}
\begin{column}{0.7\twocolwid}
    \renewcommand*{\bibfont}{\footnotesize}
    \printbibliography
\end{column}
\begin{column}{0.3\twocolwid}
    \setbeamercolor{block alerted title}{fg=black,bg=norange} % Change the alert block title colors
    \setbeamercolor{block alerted body}{fg=black,bg=white} % Change the alert block body colors
    \begin{alertblock}{Contact Information}
        \begin{figure}
            \captionsetup{labelformat=empty}
            \begin{subfigure}{0.45\textwidth}
                \centering
                \captionsetup{labelformat=empty}
                \includegraphics[width = 0.9\linewidth]{\plotpath/qr-paper.png}
                \caption{\textbf{Paper}}
            \end{subfigure}
            \begin{subfigure}{0.45\textwidth}
                \centering
                \captionsetup{labelformat=empty}
                \includegraphics[width = 0.9\linewidth]{\plotpath/qr-website.png}
                \caption{\textbf{Website}}
            \end{subfigure}  
        \end{figure}
    \end{alertblock}
\end{column}
\end{columns}

\end{column}  % End of outer 2nd columns
\end{columns} % End of all the columns in the poster

\end{frame} % End of the enclosing frame

\end{document}
