\documentclass[pdf]{beamer}

\usetheme{Berlin}

\usepackage[utf8]{inputenc}
\usepackage{hyperref}

% Math
\usepackage{amsmath, amssymb, mathtools, bbm, bm}
\usepackage{xcolor-solarized}


\let\d\relax
\newcommand{\d}[1]{\mathrm{d}#1}

\newcommand{\control}[1]{\bm{#1}}

% Diagrams
\usepackage{tikz} 
\usepackage{tikzit}
\usepackage{tikzscale}
\usepackage{pgfplots}
\usepackage{graphicx}
\usepackage{wrapfig}
\usepackage{relsize}


% Graphs
\usepackage{tikz} 
\usepackage{tikzit}
\usepackage{pgfplots}
\usepackage{ifdraft}
\usetikzlibrary{positioning,fit,calc,decorations.pathreplacing,calligraphy, arrows.meta, pgfplots.groupplots}

\pgfplotsset{compat=1.18}

\newcommand{\inputtikz}[1]{%  

    \ifdraft{
    \includegraphics[width=0.5\textwidth]{example-image-a}
    }{
    \IfFileExists{#1}{
        \input{#1}
        }{
        \includegraphics[width=0.5\textwidth]{example-image-a}
    }
    }
}
\pgfplotsset{compat=1.18}

\usepackage[bibencoding=utf8, style=apa, backend=biber, eprint=false]{biblatex}
\addbibresource{../../proposal/scc-tipping-points.bib}

\author{Andrea Titton}
\title[Climate Tipping Points and Optimal Emissions]{\small Climate Tipping Points and\\ Optimal Emissions}
\institute{CeNDEF, University of Amsterdam}
\date{PhD seminar, 11/09}

\begin{document}

\frame[plain]{\titlepage}

\section{Motivation}
\begin{frame}
    \begin{enumerate}
        \item To get optimal emissions, climate dynamics matter (\cite{dietz_are_2020, dietz_economic_2021})
        \pause \item Large focus on stochastic tipping point, via jump processes (\cite{van_den_bremer_risk-adjusted_2021,lin_social_2023})
        \pause \item In climate models most tipping points are caused by bifurcations (\cite{ashwin_tipping_2012,ashwin_extreme_2020})
    \end{enumerate}
\end{frame}

\begin{frame}
    In this paper I look at one such tipping point: \textbf{ice-albedo feedback}

    \pause \begin{figure}
        \centering
        \includegraphics[width = 0.5\linewidth]{../figures/feedback.png}
    \end{figure}
\end{frame}


\begin{frame}
    \centering
    \begin{figure}
        \includegraphics[width = 0.9\paperwidth]{../figures/sinkhole.png}
    \end{figure}
\end{frame}


\section{Climate Model}

\begin{frame} \frametitle{Temperature}
    \begin{columns}
        \column{0.3\linewidth}
        \begin{enumerate}
            \item ${\color{orange} S}$ solar radiation
            \item ${\color{red} T}$ temperature
        \end{enumerate}
        \column{0.6\linewidth}
        \begin{equation*}
            {\color{orange} S} = \eta \sigma {\color{red} T}^4
        \end{equation*}
    \end{columns}
\end{frame}

\begin{frame} \frametitle{Temperature}
    \begin{columns}
        \column{0.3\linewidth}
        \begin{enumerate}
            \item ${\color{orange} S}$ solar radiation
            \item ${\color{red} T}$ temperature
            \item ${\color{blue} M}$ carbon concentration
        \end{enumerate}
        \column{0.6\linewidth}
        \begin{equation*}
            {\color{orange} S} = \eta \sigma {\color{red} T}^4 - (G_0 + G_1 \log {\color{blue} M} / M^{\mathrm{p}})
        \end{equation*}
    \end{columns}
\end{frame}

\begin{frame} \frametitle{Temperature}
    \begin{columns}
        \column{0.3\linewidth}
        \begin{enumerate}
            \item ${\color{orange} S_0} (1 - \lambda({\color{red} T}))$ solar radiation
            \item ${\color{red} T}$ temperature
            \item ${\color{blue} M}$ carbon concentration
        \end{enumerate}
        \column{0.6\linewidth}
        \begin{equation*}
            {\color{orange} S_0} (1 - \lambda({\color{red} T})) = \eta \sigma {\color{red} T}^4 - (G_0 + G_1 \log {\color{blue} M} / M^{\mathrm{p}})
        \end{equation*}
    \end{columns}
\end{frame}


\begin{frame} \frametitle{Temperature}
    \begin{columns}
        \column{0.3\linewidth}
        \begin{itemize}
            \item ${\color{orange} S_0} (1 - \lambda({\color{red} T}))$ solar radiation
            \item ${\color{red} T}$ temperature
            \item ${\color{blue} M}$ carbon concentration
        \end{itemize}
        \column{0.6\linewidth}
        \begin{equation*}
            \begin{split}
                \epsilon \; \d {{ \color{red} T }} = \big( &{\color{orange} S_0} (1 - \lambda({\color{red} T})) - \eta \sigma {\color{red} T}^4 \\ 
                + &G_0 + G_1 \log {\color{blue} M} / M^{\mathrm{p}} \big) \d t \\
                + &\sigma_T \d W
            \end{split}
        \end{equation*}
    \end{columns}
\end{frame}

\begin{frame} \frametitle{Albedo loss}
    \begin{equation*}
        \lambda({\color{red} T}) = \lambda_1 - (1 - L({\color{red} T})) \Delta \lambda
    \end{equation*}
    \begin{figure}
    \resizebox{!}{0.4\linewidth}{
        \inputtikz{../../../plots/albedo.tikz}
    }
    \end{figure}
\end{frame}

\begin{frame} \frametitle{Albedo loss}
    \begin{equation*}
        {\color{orange} S_0} (1 - \lambda({\color{red} T})) = \eta \sigma {\color{red} T}^4 -  G_0 - G_1 \log {\color{blue} M} / M^{\mathrm{p}}  
    \end{equation*}
    \begin{figure}
    \resizebox{!}{0.4\linewidth}{
        \inputtikz{../../../plots/nullcline.tikz}
    }
    \end{figure}
\end{frame}

\begin{frame} \frametitle{Carbon Concentration}
    \begin{columns}
        \column{0.3\linewidth}
        Let $\gamma^\mathrm{b}$ be the growth rate of carbon concentration $M$ under the business-as-usual scenario.
        \column{0.6\linewidth}
        \begin{figure}[H]
            \inputtikz{../../../plots/growthmfig.tikz}
        \end{figure}
    \end{columns}
\end{frame}


\begin{frame} \frametitle{Carbon Concentration}
    \begin{columns}
        \column{0.3\linewidth}
        \begin{equation*}
            \d m = \frac{\d M}{M} = \left( \gamma^\mathrm{b} - \control{\alpha} \right) \d t
        \end{equation*}
        \column{0.6\linewidth}
        \begin{figure}[H]
            \centering
            \inputtikz{../../../plots/growthmfig.tikz}
        \end{figure}
    \end{columns}
\end{frame}

\begin{frame}{Business-as-usual dynamics}
    \begin{figure}[H]
        \centering
        \inputtikz{../../../plots/presentation/baufig.tikz}
    \end{figure}
\end{frame}

\section{Economy}

\begin{frame}{$AK$ Economy (\cite{hambel_optimal_2021})}
    \begin{columns}
        \column{0.3\linewidth}
        \begin{itemize}
            \item ${\color{cyan} y}$ log-output
            \item ${\color{purple} k}$ log-capital
            \item $I$ investment in capital
            \item $B$ abatement expenditure
            \item $C$ consumption
        \end{itemize}
        \column{0.6\linewidth}
        \begin{align*}
            {\color{cyan} y} &= \log(A) + {\color{purple} k} \text{ s.t.} \\
            {\color{cyan} Y} &= I + B + C
        \end{align*}
    \end{columns}
\end{frame}

\begin{frame}{Climate change slows growth}
    \begin{columns}
        \column{0.3\linewidth}
        \begin{itemize}
            \item ${\color{purple} k}$ log-capital
            \item ${\color{red} T}$ temperature
            \item $I$ investment in capital
            \item $B$ abatement expenditure
        \end{itemize}
        \column{0.6\linewidth}
        \begin{equation*}
            \d {{\color{purple} k}} = \left(\frac{I}{{\color{purple} K}} - \delta_k({\color{red} T}) - \frac{\kappa}{2} \left( \frac{I}{{\color{purple} K}}  + \frac{B}{{\color{purple} K}}\right)^2\right) \d t
        \end{equation*}
    \end{columns}
\end{frame}


\begin{frame}{Climate damages}
    \begin{figure}[H]
        \centering
        \inputtikz{../../../plots/presentation/damagefig.tikz}
    \end{figure}
\end{frame}

\begin{frame}{Making everything a rate...}
    \begin{columns}
        \column{0.3\linewidth}
        \begin{itemize}
            \item ${\color{cyan} Y}$ output
            \item $B$ abatement expenditure
            \item $C$ consumption
            \item $\mathbf{\chi}$ consumption rate
            \item $\mathbf{\varepsilon}$ emissivity rate
        \end{itemize}
        \column{0.6\linewidth}
        \begin{equation*}
            \mathbf{\chi} \coloneqq \frac{C}{{\color{cyan} Y} } \text{ and } \frac{\omega}{2} \mathbf{\varepsilon}^2 \coloneqq \frac{B}{{\color{cyan} Y}}
        \end{equation*}
    \end{columns}
\end{frame}

\begin{frame}{Emissivity rate $\varepsilon$ to abatement $\alpha$}
    \begin{columns}
        \column{0.3\linewidth}
        \begin{itemize}
            \item ${\color{blue} M}$ CO$_2$ concentration
            \item $E^{\mathrm{B}}$ BaU emissions
            \item $\delta_m({\color{blue} M})$ natural decay of CO$_2$
            \item $\gamma^{\mathrm{B}}$ BaU growth of ${\color{blue} M}$
        \end{itemize}
        \column{0.6\linewidth}
        \begin{equation*}
            \varepsilon = 1 - \frac{{\color{blue} M}}{ E^{\mathrm{B}}} \left( \delta_m({\color{blue} M}) + \gamma^{\mathrm{B}} - \mathbf{\alpha} \right)
        \end{equation*}
    \end{columns}
\end{frame}


\begin{frame}{Putting it all back into capital...}
    \begin{equation*}
        \frac{\d {{\color{purple} k}}}{\d t} = \overbrace{A (1 - \mathbf{\chi}) - \frac{A\kappa}{2} (1 - \mathbf{\chi})^2}^{\text{Standard consumption problem $\phi(\chi)$}} - \underbrace{\frac{A \omega}{2} \mathbf{\varepsilon}^2}_{\text{abatement}} - \overbrace{\delta_k({\color{red} T})}^{\text{climate change}}
    \end{equation*}
\end{frame}


\begin{frame}{But we really care about output...}
    \begin{equation*}
        \frac{\d {{\color{cyan} y}}}{\d t} = \overbrace{\varrho + \phi(\chi)}^{\text{Economic growth}} - \underbrace{\frac{A \omega}{2} \mathbf{\varepsilon}^2}_{\text{abatement}} - \overbrace{\delta_k({\color{red} T})}^{\text{climate change}}
    \end{equation*}
\end{frame}

\section{Optimal Emissions}

\begin{frame}{Social Planner}
    The social planner at time $\tau$ is trying to maximise \begin{equation*}
        U(\tau, \alpha, \chi) = \mathbb{E} \int^\infty_\tau f(C(t), U(t, \alpha, \chi)) \d t
    \end{equation*} where $f(C, U)$ is the Epstein-Zin aggregator.
\end{frame}

\begin{frame}{Why Epstein-Zin?}
    \begin{columns}
        \column{0.5\linewidth}
        I flip a coin today. You get $2$€ every day if it is head and $1$€ every day if it is tails.
        \column{0.5\linewidth}
        Each day I flip a coin. That day you get $2$€ if it is head and $1$€ if it is tails.
    \end{columns}
\end{frame}

\begin{frame}{Value Function}
    Looking for $V(t, \chi, \alpha) = \sup_{\chi, \alpha} U(t, \alpha, \chi)$, which satisfies
    \begin{equation*}
        -\partial_t V = f(\chi {\color{cyan} Y}, V) + \nabla V \cdot (\text{drift of state}) + \partial^2_T V \; \sigma^2_T
    \end{equation*}
\end{frame}


\begin{frame}
    We need to find functions $\alpha, \chi, V$ from a four dimensional state space $t, T, M, y$ to $\mathbb{R}$. We must do this numerically but how?
    
    \hspace{2em}

    \pause No really... how?
\end{frame}

\section{Conclusion}
\begin{frame}{Conclusion}
    \begin{enumerate}
        \item Tipping points in the climate systems are (mostly) induced by bifurcations
        \pause \item Optimal emissions paths matter, not only total emissions (would you walk on the edge of a ditch)
        \pause \item TODO: solve this HJB numerically?  
    \end{enumerate}
\end{frame}

\begin{frame}{Thank you!}
    \centering
    \begin{figure}
        \includegraphics[width = 0.7\linewidth]{../figures/netherlands.png}
        \caption{Ed Hawkins, Berkeley Earth data}
    \end{figure}
\end{frame}



\end{document}
